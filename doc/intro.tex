\addchap{Introduction}

%Most IT infrastructures were not built to support the explosive growth in computing capacity and information that we see today. Many data centers have become highly distributed and somewhat fragmented. As a result, they are limited in their ability to change quickly and support the integration of new types of technologies or to easily scale to power the business as needed.

%When equipped with a highly efficient, shared, and dynamic infrastructure, along with the tools needed to free up resources from traditional operational demands, IT can more efficiently respond to new business needs. As a result, organizations can focus on innovation and on aligning resources to broader strategic priorities. Decisions can be based on real-time information.

Les centres de traitement de données\index{Data Centre!Centre de traitement de données} évoluent aujourd'hui à un rythme intense pour accompagner l'explosion constatée dans l'utilisation des données. L'accélération de l'innovation dans l'informatique impose une rénovation constante du business. La virtualisation a permis aux centres de données d'améliorer la productivité de ses serveurs, mais pour arriver à l'agilité souhaitée, les data centres doivent faire évoluer leurs réseaux et débloquer le Cloud Computing. Cette étude analyse les applications \gls{sdn} pour distinguer ses apports dans le contexte des data centres et habiliter le passage au Cloud Computing.

\par 
La plupart des infrastructures de \gls{ti} n'ont pas été construites pour supporter la croissance explosive de la capacité de traitement de l'information observé aujourd'hui. Plusieurs centres de données sont devenus hautement distribués et relativement fragmentés. Ils se trouvent donc limités dans leur capacité à évoluer rapidement et à supporter l'intégration des nouveaux types de technologies ou se mettre à l'échelle pour monter le business en puissance selon les besoins de ses consommateurs.

\par 
Lors qu'ils sont équipés d'infrastructures performantes, partagées et dynamiques ainsi qu'avec des outils nécessaires pour libérer les ressources de la demande traditionnelle, les \gls{si} peuvent alors répondre plus efficacement aux besoins métiers. Par conséquent, les structures pourraient se focaliser dans l'innovation et ajuster des ressources à leurs priorités stratégiques. Cela soulagerait la prise de décisions, qui pourrait se concentrer sur l'information en temps-réel.

\par
Alors que le coût du réseau dans un data centre est estimé à 15\% \cite{cloudCosts} du total, sans être un des plus élevés, il est largement établi qu'il représente un élément clé pour la réduction des coûts et l'augmentation du retour sur investissement. Les coûts d'investissement dans les serveurs ont été évalués à 45\% des coûts des data centres. Malheureusement la charge utile des serveurs est remarquablement basse, arrivant à seulement 10\% d'utilisation dans certains exemples.

\par 
La technique de la virtualisation a permis le déplacement des processus entre machines, mais des contraintes réseau continuent à limiter l'agilité dans les data centres. L'agilité est définie par la capacité d'affecter tout service n'importe où dans le data centre, tout en assurant la sécurité, la performance et l'isolation entre tous les services. Les designs des réseaux conventionnels dans un data centre empêchent cette agilité ; par nature ils fragmentent  à la fois les réseaux et la capacité des serveurs, limitant et réduisant la croissance dynamique des pools de serveur. \cite{cloudCostsAgility}



\par 
L'agilité est donc un élément clé ; certaines entreprises s'évertuent à déployer des nouvelles applications ou faire évoluer les existantes au rythme de la croissance de leur business. Selon le sondage mené par AlgoSec avec 240 professionnels de l'informatique, 25\% des organisations participantes doivent attendre plus de 11 semaines pour qu'une nouvelle application soit mise en ligne (et dans 14\%, ce temps dépasse 5 mois). Les résultats révèlent également que 59\% des entreprises nécessitent plus de huit heures pour réaliser un changement de connectivité dans une application. \cite{algoSecSurvey}


\par
Cependant, lors du passage au Cloud, les entreprises réalisent que la virtualisation des serveurs est considérablement limitée par les designs Ethernet classiques et les contrôles de sécurité réseau traditionnels. Avec l'augmentation de la virtualisation au sein des data centres, quatre tâches critiques deviennent contraignantes :
\begin{itemize}
\item Prévention de la congestion du trafic ;
\item Réduction de la complexité des politiques réseau et maintien du niveau de service;
\item Élimination des points aveugles qui conduisent à des pannes ;
\item  Scellage des failles de sécurité pour protéger les données.
\end{itemize}
%But as businesses move to the private cloud, they are finding server virtualization is severely limited by clas- sic Ethernet designs and traditional network security controls. As data center virtualization scales, four critical tasks become increasingly cumbersome:n Preventing traffic bottlenecksn Reducing complexity of network policy and service level assurancen Eliminating management blind spots that lead to outagesn Sealing up security loopholes to protect data

\par
Cette étude a pour but de démontrer comment SDN peut être appliqué aux data centres pour débloquer le Cloud Computing et dépasser ses limites. Dans le premier chapitre, le contexte des data centres sera défini. Ensuite, les enjeux dans l'aspect réseau seront exposés. Le chapitre suivant présentera les applications SDN qui répondent à ses enjeux. Enfin, le quatrième et dernier chapitre démontrera les apports de SDN dans ce cadre.

