


\chapter{L'aspect réseau}
Dans ce chapitre, les principales problématiques data centre dans un aspect réseau seront présentées et analysées.  Comment on fait aujourd'hui ? Quels sont les limites ? 


\section{Le réseau dans un data centre}
As organizations undertake information technology (IT) optimization projects, such as data center consolidation and server virtualization, they need to ensure that the proper level of focus is given to the critical role of the network in terms of planning, execution, and overall project success. While many consider the network early in the planning stages of these projects and spend time considering this aspect of these initiatives, many more feel that additional network planning could have helped their projects be more successful.


The most common types of network changes in IT optimization projects include implementing new network equipment, adding greater redundancy, increasing capacity by upgrading switches, improving network security, and adding network bandwidth. However, many network requirements associated with these changes and the overall initiative are typically not identified until after the initial stages of the project and often require rework and add unanticipated costs. Regardless of project type, network challenges run the risk of contributing to increased project time lines and/or costs.


The networking aspects of projects can be challenging and user complaints about the network are frequently heard. Important challenges include the inability to perform accurate and timely root-cause analysis, understand application level responsiveness, and address network performance issues. Simply buying more network equipment does not necessarily or appropriately address the real requirements.


Looking ahead, many expect that the network will become more important to their companies' overall success. To address this, networking investments related to support of server and storage virtualization are currently at the top of the list for consideration, followed by overall enhancement and optimization of the networking environment.
To support virtualization of the entire IT infrastructure and to continue to optimize the network, IT organizations need to make architectural decisions in the context of the existing infrastructure, IT strategy, and overall business goals.


Developing a plan for the network and associated functional design is critical. Without a strong plan and a solid functional design, networking transitions can be risky, leading to reduced control of IT services delivered over the network, the potential for high costs with insufficient results, and unexpected performance or availability issues for critical business processes.


With a plan and a solid functional design, the probability of success is raised: a more responsive network with optimized delivery, lower costs, increased ability to meet application service level commitments, and a network that supports and fully contributes to a responsive IT environment.

\section{Challenges réseau}


\section{Différents usages}

\section{Agilité}

\section{Sécurité}

\section{Changement des réseaux data centre}
In order to enable a dynamic infrastructure capable of handling the new requirements that have been presented in the previous section, a radical shift in how the data center network is designed is required.
The figures here show the comparison between the traditional thinking and the new thinking that enables this change of paradigm.
Figure 1-3 illustrates a traditional, multitier, physical server-oriented, data center infrastructure.

