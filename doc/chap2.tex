


\chapter{L'aspect réseau}
Dans ce chapitre, les principales problématiques data centre dans un aspect réseau seront présentées et analysées. Seront abordés quelques questionnements sur les réseaux classiques pour tenter de répondre aux nouveaux besoins data centres. Le chapitre présentera comment divers scénarios sont traités aujourd'hui et quels sont les limites. 


\section{Le rôle du réseau dans les projets de TI}

Lors du développement de projets pour l'optimisation en \gls{ti} tels que la consolidation de data centres et la virtualisation de serveurs, une attention spéciale doit être donnée au rôle critique des réseaux dans la planification, exécution et succès en général du projet. Il est souvent admis que des planifications supplémentaires par rapport aux réseaux auraient pu contribuer au succès de plusieurs projets.

%As organizations undertake information technology (IT) optimization projects, such as data center consolidation and server virtualization, they need to ensure that the proper level of focus is given to the critical role of the network in terms of planning, execution, and overall project success. While many consider the network early in the planning stages of these projects and spend time considering this aspect of these initiatives, many more feel that additional network planning could have helped their projects be more successful.


Les principaux types de modifications dans ces projets incluent l'implémentation d'équipement réseau supplémentaire pour augmenter ou améliorer la redondance, la capacité du réseau, la sécurité réseau et/ou la bande passante. Cependant, plusieurs requis associés à ces changements ne sont pas, en général, identifiés au tout début du projet. Très souvent ils ne sont détectés qu'après les étapes initiales du projet, imposant un supplément de travail et l'ajout de coûts non anticipés.

%The most common types of network changes in IT optimization projects include implementing new network equipment, adding greater redundancy, increasing capacity by upgrading switches, improving network security, and adding network bandwidth. However, many network requirements associated with these changes and the overall initiative are typically not identified until after the initial stages of the project and often require rework and add unanticipated costs. Regardless of project type, network challenges run the risk of contributing to increased project time lines and/or costs.


Les aspects réseau d'un projet peuvent être difficiles à gérer, et des critiques sur le fonctionnement général des réseaux sont fréquemment entendues. Des défis importants incluent la réalisation d'analyses des causes précises et opportunes, la compréhension de la réactivité au niveau applicatif et la révélation des origines de problèmes de performance. Le simple achat d'équipement réseau n'aborde pas nécessairement ou proprement les requis réels.

%The networking aspects of projects can be challenging and user complaints about the network are frequently heard. Important challenges include the inability to perform accurate and timely root-cause analysis, understand application level responsiveness, and address network performance issues. Simply buying more network equipment does not necessarily or appropriately address the real requirements.


%Looking ahead, many expect that the network will become more important to their companies' overall success. To address this, networking investments related to support of server and storage virtualization are currently at the top of the list for consideration, followed by overall enhancement and optimization of the networking environment. To support virtualization of the entire IT infrastructure and to continue to optimize the network, IT organizations need to make architectural decisions in the context of the existing infrastructure, IT strategy, and overall business goals.


Pour supporter la virtualisation complète d'une infrastructure de \gls{ti} et continuer à optimiser le réseau, des décisions sur l'architecture doivent être faites dans le contexte de l'infrastructure existante, de la stratégie et des objectifs larges du business. Sans le développement d'un plan réseau et la conception fonctionnelle associée, les transitions réseau peuvent être risquées et conduire à un contrôle réduit des services livrés, coûts potentiellement élevés contre des résultats insuffisants et des problèmes inattendus de performance ou disponibilité. 
%Developing a plan for the network and associated functional design is critical. Without a strong plan and a solid functional design, networking transitions can be risky, leading to reduced control of IT services delivered over the network, the potential for high costs with insufficient results, and unexpected performance or availability issues for critical business processes.

%The network is the natural home for management and enforcement of policies relating to risk, performance, and cost. Only the network sees all data, connected resources, and user interactions within and between clouds. The network is thus uniquely positioned to monitor and meter usage and performance of distributed services and infrastructure. Management tools for the data center and wider networks have moved from a user-centric focus (for example, GUI design) to today’s process-centric programmatic capabilities. In the future, the focus will most likely shift toward behavioral- and then cognitive-based capabilities.
%The network also has a pivotal role to play in promoting resilience and reliability. For example, the network, with its unique end-to-end visibility, helps support dynamic orchestration and redirection of workloads through embedded policy-based control capabilities. The network is inherently aware of the physical location of resources and users. Context-aware services can anticipate the needs of users and deploy resources appropriately, balancing end-user experience, risk management, and the cost of service.

Traditionnellement un plan et conception fonctionnelle solides suffisaient pour augmenter le succès des projets avec un réseau réactif, optimisé, moins cher et répondant mieux aux engagements des services applicatifs. Alors que ce plan reste essentiel, il est difficile dans le contexte d'utilisation actuel de maitriser complètement  à l'avance la charge, dimension et tout autre requis qui doivent être assurés par les réseaux.

En face à ses difficultés et aux critiques reçues, il est facile  de finir par interpréter les réseaux comme un élément bloquant pour le succès des projets. Même si sa gestion et planification peuvent être complexes, les réseaux représentent un moyen naturel pour le management et renforcement des politiques liées aux risques, performance et coûts. Seulement le réseau voit toutes les données, ressources connectées et interactions des utilisateurs à travers le Cloud. Le réseau est donc positionné de manière unique pour surveiller et mesurer l'usage et la performance des services distribués et de l'infrastructure. Les réseaux ont également un rôle central pour favoriser l'extensibilité et disponibilité. Par exemple avec sa vue unique bout-en-bout, les réseaux peuvent détecter la charge et dynamiquement la rediriger selon les politiques de contrôle.

Afin d'atteindre ce niveau de management et orchestration des ressources, on cherche  à concevoir des architectures réseaux qui puissent s'étendre ou se rétracter ainsi que supporter des nouveaux services dynamiquement et rapidement  en fonction des besoins immédiats. \cite{ibmPlanningVirtCCchap4} \cite{cloudAutomation} \cite{hpCloudEffectsOnNetworkIntro}
%With a plan and a solid functional design, the probability of success is raised: a more responsive network with optimized delivery, lower costs, increased ability to meet application service level commitments, and a network that supports and fully contributes to a responsive IT environment.


\section{L'architecture réseau d'un data centre typique}

La majorité des data centres d'aujourd'hui ont une structure réseau hiérarchique à trois niveaux : couche d'accès, agrégation/distribution et cœur (figure \ref{legacy_archi}). La couche accès inter-connecte tous les ressources partagés tels que serveurs, dispositifs de stockage et applications. Les réseaux d'agrégation (ou distribution) doivent fournir une haute bande passante pour la communication entre multiples réseaux d'accès. Le cœur du réseau est l'interface vers l'extérieur du réseau, qui peut être le lien avec des réseaux WAN, mobiles, VPNs ou autres types d'accès internet.  Les commutateurs composant le cœur du réseau contiennent la grande majorité de l'intelligence du réseau. 
%The access network provides connectivity to all shared enterprise servers, applications, storage devices, and any IP or office automation devices required in the data center facility. Most data center access switches are deployed at the top of the rack or at the end of the row of server racks.

%Most data centers today have a three- or four-tier hierarchical networking structure. It consists of access layer switches, aggregation switches, and core switches (Figure 1). Three-tier networking architectures were designed around client-server applications and single-purpose application servers. Client-server applications caused traffic to flow primarily in North/South (N/S) patterns: from a server up to the data center core, to the campus core where it moves out to the campus-wide network or internet. These large core switches usually contain the vast majority of the intelligence in the network.


\begin{figure}[h]
\begin{center}
\includegraphics[width=0.8\textwidth]{images/LegacyNetworkArchitecture} 
\caption{Architecture réseau typique à trois niveaux. \cite{cloudReadyNetworkJuniper}} \label{legacy_archi}
\end{center}
\end{figure}



%The core network provides a fabric for high-speed packet switching between multiple access network devices. Due to their location in the network, core-layer switches must provide scalable, high-performance, high-density, wire-rate ports, and HA hardware and software features that deliver carrier-class reliability and robustness. The core serves as the gateway where all other modules such as the WAN edge meet. It typically requires a 10GbE interface for high-level throughput, and maximum performance to meet oversubscription levels. The core provides high-speed throughput for all data going into and out of the data center, and it must provide resilient, fail-safe Layer 3 connectivity to multiple access layer devices.


%The edge network provides the communication links to end user networks of various types. These can be private WAN or campus backbones, mobile access networks, VPNs, or other types of Internet access. The high performance and reliability of these connections improve user experience. Agility ensures that users will have access to applications and services where and when they are needed. In addition, multilayered security controls ensure that users, applications and data are protected at appropriate levels.

Cette architecture a été conçue pour les applications client-serveurs dans des serveurs applicatifs dédiés. Dans cette conception le flux du trafic se fait essentiellement à partir des serveurs en direction le cœur du réseau et ensuite dehors vers internet ou autre (flux Nord/Sud).

Ce modèle d'architecture a été la principale référence de topologie réseau pour les data centres. Cependant, il devient compliqué à le maintenir dans le contexte du Cloud Computing lors que les applications actuelles commencent à être plus distribuées, avec plusieurs couches et orientées livraison de services. Ces changements des applications ont impacté les réseaux en ce qui concerne le volume et les flux du trafic. 

Le trafic réseau interne a augmenté avec la croissance du nombre d'applications et services déployés ; l'utilisation de systèmes de fichiers distribués avec des données stockées séparément a contribué pour la perception d'une charge réseau plus importante à travers le data centre. De ces faits, on constate une forte tendance de communications inter-serveurs ou inter-VMs, impliquant un flux Est/Oust plutôt que Nord/Sud. L'industrie estime même que 80\% du trafic des applications Cloud Computing constitue des flux Est/Ouest.

%Industry sources attribute up to 80 percent of network traffic for these next generation applications coming from E/W traffic flows.

%This network architecture, however, is becoming problematic for the data center. Today’s application environments are more distributed, often with multiple tiers, and oriented toward service delivery. These application architecture changes have resulted in:
%• Greater traffic volume on the Ethernet network, including storage traffic such as FCoE and iSCSI
%• More storage traffic as applications use distributed file systems and increase the amount of synchronization and replication data across the network
%• Greater traffic flow between peer servers such as server-to-server or virtual machine-to-virtual machine—that is, East/West (E/W) rather than primarily N/S traffic flows.

Cette architecture révèle une série d'interfaces de connexions des serveurs à divers type de réseaux tels que réseaux local, de stockage, communication entre processus. Cette disposition ajoute de la complexité et des coûts sous forme de câblage, nombre de ports, extensibilité, énergie, refroidissement etc. 

Les data centres d'aujourd'hui exigent des réseaux qui sont agiles et flexibles pour pouvoir réagir rapidement aux changements et garantir une livraison efficace de services. Par exemple, des serveurs virtuels peuvent être déplacés d'une part à autre avec une simple commande en fonction de la demande. Par conséquence, les réseaux doivent s'adapter rapidement à ces changements pour éviter des perturbation du service. Cela impose que les dispositifs semblent être connectés au même réseau local, indépendamment leur proximité physique. Cette architecture fait défaut de la flexibilité nécessaire pour effectuer ces types de changements, impactant défavorablement l'efficacité opérationnelle de tout le data centre.
%today’s data center, demands networks that are nimble and adaptable enough to react quickly to changes and maintain efficient delivery of mission critical services. Virtual servers, for example, can be moved from one part of the data center to another with a simple command to better meet demand, and networks must adapt quickly to these changes or risk service disruption. Such changes require that devices appear to be seamlessly connected on the same laN, regardless of their physical proximity to each other. legacy network architectures lack the flexibility to react to these types of changes, and this adversely affects the operational efficiency of the entire data center.
%In addition to the way boxes are physically connected and oriented, keeping the network running becomes a challenge  when existing technologies prevent network engineers from doing their jobs efficiently. everyday tasks such as monitoring devices, troubleshooting, configuration management, and software upgrades become increasingly difficult as the number of independent devices in the network increases.
%Such operational challenges are further compounded if these devices are running different versions of software or have different configurations, since software must be carefully managed across devices to ensure consistent functionality and limit exposure to bugs or other vulnerabilities. Special training or expertise may also be needed to support these configurations. as a result, the manpower required to adequately operate, maintain, and troubleshoot the unique requirements of each network device can be enormously time-consuming and costly.


Afin de mieux visualiser ces difficultés, les sections subséquentes vont détailler quelques scénarios critiques pour les réseaux traditionnels. Ces scénarios supposent des infrastructures basées l'architecture à trois niveaux présentée et l'objectif de fonctionner en mode Cloud Computing. \cite{hpCloudEffectsOnNetworkChanging} \cite{cloudReadyJuniperReferenceNetworkInfra}  \cite{cloudReadyNetworkJuniper} \cite{bigDataBookChap4}

\section{Applications business en transformation}

Avec multiples VMs exécutant dans le même hôte, partageant une carte réseau unique au moyen d'un switch logiciel (ou virtuel), les application ont donc moins de ressources réseau que quand elles sont dédiées un serveur physique. Cela peut conduire à des problèmes de performance réseau comme bande passante réduite et temps de latence augmenté et même à d'autres difficultés mineures comme plages d'adresses IP disponible ; les applications peuvent ne pas être en mesure de traiter ces questions.


%With a physical machine running a network application, that application can have access to the full resources of the network card. Once 10 virtual machines are running together on a single host—sharing a single hardware network card and streaming together through a software switch—that same virtualized application now has fewer networking resources available than it did before. This can lead to overall network performance issues, reduced bandwidth, and increased latency; all issues the application might not be able to deal with. Even smaller issues such as IP address availability can be impacted by virtualization sprawl.


Les plateformes d'infrastructure virtuelle typiquement fournissent des logiciel pour migrer les instances de VMs actives d'un dispositif physique à l'autre; VMware Distributed Resource Scheduler (DRS) et VMotion sont des exemples de ce type de solution. Ces solutions ne souvent pas au courant de l'état des applications ou du réseau. Par exemple, une VM peut être migrée au milieu d'une transaction bancaire sans prendre en compte l'état de persistance de l'opération, le nombre des connexions ou la charge réseau que la VM est en train de traiter. Cela peut causer des transactions non réussites, perte de données et problèmes plus graves avec les utilisateurs. Par ailleurs, ces technologies permettent de migrer une VM vers un serveur physique avec plus de charge CPU disponible, mais elles n'ont pas d'information sur la capacité réseau dans ce serveur.  
%Virtual infrastructure platforms typically include software that can migrate live virtual machine instances from one physical device to another; VMware Distributed Resource Scheduler (DRS) and VMotion are examples of live migration solutions. Like basic OS virtualization, these migration tools are unaware of the application state, and also have no insight into the Application Delivery Network. For example, VMotion may move a virtual machine running a web shopping cart application from one host server to another without taking into consideration the current persistent state of user’s carts, how many connections are coming into the shopping cart, or the network load that this particular virtual machine is currently handling. This migration can cause a lapse in availability; connections to the application and application persistence for the user can be lost during this live migration, resulting in failed transactions, lost shopping cart data, and frustrated users. And while VMotion attempts to move the live image to a host server that is under less load, VMotion doesn’t measure the network load of that host device. It might move a live image to a machine with more available CPU cycles but less available network capacity.



Le déplacement de charges dynamiquement exige également que les VMs restent dans un VLAN commun, dans le même réseau au niveau 2.  Pour pour voir déplacer un réseau en dehors de son domaine niveau 2, il est nécessaire d'utiliser des procédures manuelles comme attribution d'adresses IP et mise à jour des entrées DNS pour les services déplacés. Pour maximiser cette flexibilité, des technologies émergent pour élargir le domaine des réseau niveau 2.

%Moving workloads dynamically requires VMs to stay within a common VLAN in the same Layer 2 (L2) network. If you want to move a VM outside its L2 domain, you have to use manual processes such as assigning and updating the IP addresses for the failed-over services and updating DNS entries correctly. To provide maximum VM flexibility, many enterprises are evaluating ways to enlarge their L2 networks.

Des nouveaux moyens tels que  \gls{vxlan} et \gls{nvgre} étendent les réseaux couche 2 avec des réseaux couche 3. Même si cette capacité devient possible avec ces technologies, le trafic local aura toujours une meilleure performance et une latence inférieure s'ils restent dans un réseau niveau 2 plus grand.


%New capabilities such as Virtual eXtensible LAN (VXLAN) and Network Virtualization using Generic Routing Encapsulation (NVGRE) logically extend an L2 network across L3 networks. However, even with this potential to move VMs across a L3 network, local traffic will still have higher performance and lower latency if it stays within a large L2 network.

\subsection{Aspect Multi-tenant}

Cet exemple traite l'aspect multi-tenant des infrastructures Cloud. Pour simplifier, on assume une configuration avec deux réseaux : une pour le groupe d'ingénieurs et l'autre pour les ventes. 

Pour réaliser cela avec les technologies réseaux traditionnelles, on utilise le concept de \gls{vlan}. Le réseau ingénieurs est, par exemple affecté, au VLAN-1 et le VLAN-2 pour la vente. Cette attribution doit être réalisée par le système de management Cloud des switchs virtuels dans l'hyperviseur. Ensuite, la configuration doit être réalisée dans tous les switches du réseau data centre et dans le switch auquel le routeur est connecté. Quand le VLAN est déjà utilisé, l'interconnexion n'est pas possible, cela veut dire qu'une administration continue des VLANs attribués doit être mise en place. Tout cela est fait de manière fastidieuse et manuelle par l'opérateur du réseau.

Chacun des réseaux est connecté à un routeur, qui doit être configuré pour prendre en compte ces VLANs. Cela veut dire que le trafic doit toujours monter et descendre pour passer par ce router (réseau accès et de distribution), ce qui n'est pas très performant. La capacité du routeur peut limiter les communication inter-départements. 

Pour fournir des adresses IP automatiquement aux applications clients, un serveur DHCP doit être attribué à chaque réseau. L'équipe d'opération réseau met en place un serveur DHCP pour chaque réseau en accord avec les configurations dans les routeurs.

Le routeur doit implémenter les règles de sécurités pour permettre le trafic d'applications business entre les deux départements et l'accès internet. Les responsables pour la sécurité doivent appliquer les politiques dans les interfaces du routeur pour assurer que seulement les flux de trafic permis sont transférés.

Cet exemple illustre la complexité opérationnel pour la configuration d'un réseau afin de supporter une application. L'intervention exige un haut niveau de manipulation manuelle et concerne différents éléments de l'architecture. La totalité de la procédure est complexe et susceptible à divers erreurs. Tout changement associé, comme le déplacement d'un serveur, extension d'un des réseaux, modification de la configuration des tenant, implique quasiment la répétition complète du procédé et sa ré-validation.

Dans le cas ou la situation décrite doit être réalisée pour multiples consommateurs simultanément, il y aura clairement un grand délai de d'implémentation. Dans un contexte Cloud, l'accommodation de demandes réseau durant plusieurs jours ou semaines devient critique et inacceptable. 


\subsection{Interconnexion WAN}

Ce scénario illustre la connexion des réseaux teants au WAN pour accès aux sites distants. Dans l'approche traditionnelle, un VLAN doit être créé entre le routeur du data centre et le routeur WAN


%The example assumes a customer tenant configuration with two networks: one to be used by the engineering group, and the other by sales. The engineering department can only communicate with the sales department through web traffic; all other communication should be blocked. Both the engineering and sales teams are allowed to communicate with the Internet. 

Un protocole de routage doit être défini pour fournir une résistance en cas de failles de communication. La configuration doit être appliqué dans les deux routeurs impliqués. Pour offrir des mécanismes anti-failles aux niveaux 2 et trois dans le WAN, plusieurs points VPNs doivent être déployés et configurés, résultant en une complexité additionnelle pour l'opération.

Traditionnellement,  les réseaux du data centre et du WAN sont gérées  par des responsables distincts. Cela implique la coordination entre les deux personnels réseaux, en général via une structure formelle de projet et procédures manuelles. Par ailleurs, avec les demandes de divers tenant, ce modèle introduit des délais ainsi que des problèmes d'extensibilités supplémentaires lors qu'ils requièrent des VLAN ou protocoles de routage dédiés.



%\section{Différents usages}




%\section{Agilité}


\section{Aspects de Sécurité}

Comme développé précédemment, actuellement on exige que les applications livrent des informations et services spécifiques au contexte immédiatement, à une latence réduite et à une haute performance. Au même temps, le Cloud Computing et les applications orientées services introduisent des demandes plus strictes au niveau service. La mutualisation de ressources et les infrastructures multi-tenantes ont apporté des nouvelles préoccupations sur la sécurité qui n'existait pas précédemment et qui ne sont donc pas traitées dans l'architecture traditionnelle. Cette section a pour but d'aborder quelques menaces et question de sécurités introduites par la virtualisation et le Cloud Computing.

%We expect web applications to deliver integrated, context-specific information and services. And, we expect it right now—low-latency, high performance connections are critical. At the same time, cloud computing and service-oriented applications are introducing more stringent service-level and security demands.



%Some security risks unique to a virtualization infrastructure include communication blind spots, inter-VM attacks, and mixed trust level VMs. Instant-on gaps and resource contention are also important considerations. This section addresses each of these threats and issues.



%communication Blind Spots
\textbf{Points aveugles de la communication}

Les applications de sécurité réseaux traditionnelles ne voient pas la communication entre VMs au sein du même hyperviseur, à moins que toutes leurs communications soient routées à l'extérieur de la machine hôte vers l'application de sécurité et re-routées à l'intérieur. Cette manière de traiter la problématique introduit un considérable ralentissement du réseau. 
%In virtualized environments, traditional network security appliances are blind to the communication between VMs on the same host unless all communications are routed outside the host machine to this separate appliance. But this security configuration introduces significant time lags. 

Dans le Cloud Computing, le moyen de traiter ces deux problèmes est d'intégrer des modules de sécurité dans chaque VM pour qu'elle puissent s'auto-protéger. Cette solution obliger la re-configuration de tous les systèmes en cas de mise-à-jour des politiques de sécurité. Il manque dans l'architecture actuelle un contrôleur centra qui pourrait propager les politiques pour tous les nœuds. 
%Une manière de traiter ces deux problèmes est de placer une VM dédiée à la sécurité dans l'hôte qui gère la communication entre plusieurs VMs. La VM dédiée à la sécurité est intégrée à l'hyperviseur pour
%One way to eliminate blind spots while reducing time lags is to place a dedicated scanning security VM on the host that coordinates communication between VMs. This solution works well in a virtualized environment. However, a dedicated security VM is not ideal for a cloud environment. The dedicated security VM integrates with the hypervisor to communicate with other guest VMs. In some cloud environments, such as in a multi-tenant public cloud, users do not have access to the hypervisor. In the cloud, protection is best provided as self-defending VMs. Protection is self contained on each VM and does not require communication outside of the VM to remain secure.

%inter-Vm attacks and hypervisor compromises
\textbf{Attaques entre VMs et exposition de l'hyperviseur}

Les serveurs virtualisés utilisent les même systèmes d'exploitation et applications que le serveurs physiques. Les vulnérabilités retrouvées dans ces systèmes représentent donc des menaces aux systèmes physiques ainsi qu'aux environnements virtuels. De cette manière, quand un élément de l'environnement virtuel est compromis, l'ensemble du système est risque si un système de sécurité conscient de la virtualisation n'est pas en place. 
%Virtualized servers use the same operating systems, enterprise applications, and web applications as physical servers. Hence, the ability of an attacker to remotely exploit vulnerabilities in these systems and applications is a significant threat to virtualized environments as well. And once an attacker compromises one element of a virtual environment, oather elements may also be compromised if virtualization-aware security is not implemented.

Dans ce scénario, un hacker peut attaquer un système invité, qui peut ensuite infecter d'autres VMs. Le risque augmente avec le nombre de VMs hébergées. Une protection capable de détecter des activités malicieuses au niveau des VMs doit être mise en place, indépendamment de la localisation des VMs dans l'environnement virtuel.
%In one scenario, an attacker can compromise one guest VM, which can then pass the infection to other guest VMs on the same host. Co-location of multiple VMs increases the attack surface and risk of VM-to-VM compromise. A firewall and an intrusion detection and prevention system need to be able to detect malicious activity at the VM level, regardless of the location of the VM within the virtualized environment.

Un autre mode d'attaque implique l'hyperviseur qui devient une cible d'attaque dû à son rôle et responsabilités.Certains attaques vont essayer de traverser l'espace d'isolation des VMs pour compromettre l'hyperviseur. Sécuriser l'hyperviseur est donc indispensable, malgré la complexité à le mettre en place.
%Another attack mode involves the hypervisor, which is the software that enables multiple VMs to run within a single computer. While central to all virtualization methods, hypervisors bring both new capabilities and computing risks. A hypervisor can control all aspects of all VMs that run on the hardware, so it is a natural security target. Therefore, securing a hypervisor is vital, yet more complex than it seems.

%In an attack known as “hyperjacking,” malware that has penetrated one VM may attack the hypervisor. When a guest VM attempts this attack, it is often called a “guest VM escape” because the guest VM breaks out of, or escapes, its isolated environment and attacks the host hypervisor. Once compromised, a hypervisor can then attack other guest VMs on that host.

%VMs make requests to the hypervisor through several different methods, usually involving a specific application programming interface (API) call. An API is the interface created to manage VMs from the host machine. These APIs are prime targets for malicious code, so virtualization vendors attempt to ensure that APIs are secure and that VMs make only authentic (i.e. authenticated and authorized) requests. Because this is a critical path function, speed is a significant requirement in all hypervisors to ensure that overall performance is not impeded.
%When attackers targeted a zero-day vulnerability in a virtualization application called HyperVM made by LXLabs, as many as 100,000 web sites were destroyed [1]. In addition, certain virtualization vendors like Amazon Web Services have made their APIs public. These will undoubtedly become interesting targets for cybercriminals. Vendors that have not made their APIs public like vSphere, while not usually externally exposed, can also become potential targets for attacks within their perimeters. There is a risk that, owing to the rapid change in the API space and the current race to market, virtualization management systems will not be secure in the future.

%mixed trust level Vms
\textbf{VMs à niveaux de confiance mélangés}

VMs contenant des données sensibles peuvent finir par résider dans le même hôte qu'une autre VM moins critique, ce qui résulte dans un groupement de VMs à différents niveau de confiance. Il serait possible de séparer les VMs selon leurs niveaux de sécurité, mais ce principe contredit l'intention de la virtulisation de faire meilleure utilisation des ressources. Cette problématique implique un système de protection par VM, malgré la déjà examinée complexité à synchroniser les politiques entre tous les systèmes.
%VMs with mission-critical data may reside on the same host as VMs with less critical data – resulting in mixed trust level VMs. Enterprises can attempt to segregate these different levels of secure information on separate host machines, but in some cases, this can defeat the purpose of a virtualized environment – to make the most efficient use of resources. Enterprises need to ensure that mission-critical information is protected while still realizing the benefits of virtualization. With self-defending VM security, VMs can remain safe even in mixed trust level environments, with protection such as intrusion detection and prevention, a firewall, integrity monitoring, log inspection, and antivirus capabilities.

%instant-on gaps
\textbf{Lacunes sur les instantanées}

La virtualisation et le Cloud Computing apportent des fonctionnalités tels que approvisionnement, clonage, migration et désaffectation à la demande. Comme résultat, les VMs sont activées et inactivées à cycles rapides et assurer une sécurité consistante à ces systèmes tout en les gardant à jour peut être difficile.
%Virtualized environments are not necessarily inherently less secure than their physical counterparts. However, in some cases, the practical uses of virtualization can introduce vulnerabilities, unless administrators are aware of these vulnerabilities and take steps to eliminate them. Instant-on gaps are an example of such a vulnerability.
%Beyond server consolidation, enterprises take advantage of the dynamic nature of VMs by quickly provisioning, cloning, migrating, and decommissioning VMs as needed, for test environments, scheduled maintenance, and disaster recovery, and to support task workers who need computational resources on-demand. As a result, when VMs are activated and inactivated in rapid cycles, rapidly and consistently provisioning security to those VMs and keeping them up-to-date can be challenging.

Après une période inactivées, l'écart des VMs par rapport à la base de sécurité peut être tellement important que la simple mise sous tension peut introduire des vulnérabilités considérables. Par ailleurs, même les VMs inactives peuvent être compromises. De cette manière, nouvelles VMs peuvent être clonées à partir des templates avec sécurité dépassée et mises en ligne.
%After a period of time, dormant VMs can eventually deviate so far from the baseline security state that simply powering them on introduces significant security vulnerabilities. And even if VMs are dormant, attackers may still be able to access them. Also, new VMs may be cloned from VM templates with out-of-date security. Even when VMs are built from a template with virus protection and other security applications, the VMs need the security agent to have the latest security configurations and pattern file updates.

Quand ces VMs sont réactivées ou clonées est représentent une vulnérabilité instantanée du moment ou elles sont connectées au réseau. Une solution serait la mise en place d'une VM par hôte dédiée à mettre à jour les applications des sécurités des VMs qui sont lancées, permettant de profiter sans dangers des bénéfices de la virtualisation. 
%When dormant, reactivated, or cloned VMs have out-of-date security, attackers may be able to leverage an exploit for a longer period of time – the attack may have more longevity. Generally, if a guest VM is not online during the deployment or updating of antivirus software, it will lie dormant in an unprotected state and be instantly vulnerable when it does come online. One solution is a dedicated security VM on each host that automatically updates VMs with the latest security when activated or cloned, and safely allows enterprises to realize the benefits of virtualization.

%resource contention 
\textbf{Contention de ressources}

Quand opérations gourmandes en ressources sont appliquées aux VMs, elle peuvent rapidement sur charger le système. Par exemple, quand les balayages ou mise-à-jour des antivirus sont réalisées simultanément sur plusieurs VMs dans un même serveur physique, le système peut ne pas supporter la consommation intensive des ressources (mémoire, CPU, réseau, stockage). Cela peut impacter la performance globale des applications en exécution.
%When resource intensive operations such as regular antivirus scans and pattern file updates designed for physical environments are applied to VMs, these operations can quickly result in an extreme load on the system. When antivirus scans or scheduled updates simultaneously kick into action on all VMs on a single physical system, the result is an antivirus storm. This storm is like a run on the bank, where the bank is the underlying virtualized resource pool of memory, storage, and CPU. This performance impact hampers server applications and VDI environments.

Dans les systèmes physiques, anti-virus sont installés dans chaque système d'exploitation et consomme une mémoire additionnelle importante. L'application de cette architecture aux systèmes virtuel signifie une perte indésirable d'efficacité sur l'utilisation des ressources. Les produits qui ne sont pas conscients  de la virtualisation la randomisation ou le groupement pour éviter la charge intensif des opération simultanées.
%The legacy security architecture also results in linear growth of memory allocation as the number of VMs on a single host grows. In physical environments, antivirus software must be installed on each operating system. Applying this architecture to virtual systems means that each VM requires additional significant memory footprint  an unwanted drain on server consolidation efforts. Products that are not virtualization-aware suggest the use of randomization or grouping to avoid resource contention.

Malheureusement, la randomisation ne permet pas d'éviter les longues périodes de haute consommation du système pour les cycles complets de balayage. Le groupement ne contribue pas à la nature mobile de la virtualisation, exigeant la ré-configuration lors de la migragtion ou du clonage de VMs.
%However, randomization does not help to avoid times of high system usage and requires that a long period of time be reserved for the full scan cycle. Grouping does not allow for the mobile nature of virtualization, requiring reconfiguration when VMs are migrated or cloned.



