\documentclass[a4paper,12pt,bibliography=totoc,index=totoc,twoside,francais]{scrbook}
%\documentclass[a4paper,12pt,DIVcalc,bibliography=totoc,index=totoc,twoside,francais]{scrbook}
\KOMAoptions{titlepage,chapterprefix,open=right}
%\KOMAoptions{bibliography=totoc,index=totoc}
%\addtokomafont{chapter}{\rmfamily}
%\addtokomafont{section}{\rmfamily}

\usepackage[utf8]{inputenc}
%\usepackage[utf8x]{inputenc}
\usepackage[T1]{fontenc}
\usepackage{lmodern}
\usepackage{graphicx}
\usepackage[automark,headsepline]{scrpage2}
\usepackage[style=numeric,sorting=none,backend=biber]{biblatex}
\usepackage{csquotes}
\usepackage{xspace}
\usepackage[autolanguage]{numprint}
\usepackage{array}
\usepackage{booktabs}
\usepackage[table,svgnames,dvipsnames]{xcolor}
\usepackage[final]{pdfpages} 
\clubpenalty=5000
\widowpenalty=5000

\usepackage{graphicx,wrapfig}

\DeclareUnicodeCharacter{00A0}{ }

%\usepackage{lipsum}
%\bibliography{biblio}

\usepackage[backend=biber]{biblatex}
\addbibresource{biblio.bib}

%\usepackage{hyperref} 
\usepackage[pdfauthor={Cynthia Lopes do Sacramento}, pdftitle={Applications SDN}]{hyperref}

\usepackage{makeidx}
\makeindex

\setlength{\parskip}{1em}
\renewcommand{\baselinestretch}{1.2}


\setlength{\belowcaptionskip}{0em}
\setlength{\textfloatsep}{0em}
\setlength{\intextsep}{0em}

%\usepackage[xindy={language=french,codepage=utf8},acronym,toc=true,nonumberlist]{glossaries}
%\usepackage[acronym,toc=true,nonumberlist]{glossaries}
\usepackage[toc=true,acronym,nopostdot,nonumberlist]{glossaries}
%\usepackage[toc=true,acronym]{glossaries}

\makeglossaries


\let\Oldgls\gls%Transformation de la commande \gls en \Oldgls
\let\Oldglslink\glslink%Transformation de la commande \gls en \Oldgls
\let\Oldglspl\glspl%Transformation de la commande \gls en \Oldgls

% Création de la nouvelle commande \gls
\renewcommand{\gls}[1]{%
\textbf{\Oldgls{#1}}%
}
% Création de la nouvelle commande \gls
\renewcommand{\glslink}[2]{%
\textbf{\Oldglslink{#1}{#2}}%
}
% Création de la nouvelle commande \gls
\renewcommand{\glspl}[1]{%
\textbf{\Oldglspl{#1}}%
}

%\newacronym{rtfm}{RTFM}{Read the f\dots manual}

\newacronym{sdn}{SDN}{Software-Defined Networking : Réseau Informatique Défini par Logiciel}

\newacronym{ti}{TI}{Technologie de l'Information}

\newacronym{si}{SI}{Système d'Information}

\newacronym{nfv}{NFV}{Network Functions Virtualization, Virtualisation des fonctions réseau}

\newacronym{onf}{ONF}{Open Networking Foundation}

\newacronym{nos}{NOS}{Network Operating System, Système d'exploitation réseau}

\newacronym{vm}{VM}{Virtual Machine, Machine Virtuelle}

\newacronym{ip}{IP}{Internet Protocol, Protocole d'Internet}

\newacronym{vlan}{VLAN}{Virtual Local Area Network, Virtual LAN}

\newacronym{vxlan}{VXLAN}{Virtual eXtensible LAN}

\newacronym{gre}{GRE}{Generic Routing Encapsulation}

\newacronym{nvgre}{NVGRE}{Network Virtualization using Generic Routing Encapsulation}

\newacronym{lan}{LAN}{Local Area Network, Réseau local}

\newacronym{wan}{WAN}{Wide Area Network, Réseau étendu}

\newacronym{nat}{NAT}{Network Address Translation, Traduction d'adresse réseau}

\newacronym{dhcp}{DHCP}{Dynamic Host Control Protocol, Protocole pour la configuration automatique d'hôte}

\newacronym{dns}{DNS}{Domain Name System, Système de noms de domaine}

\newacronym{mpls}{MPLS}{MultiProtocol Label Switching, Commutation multi-protocoles par étiquettes}

\newacronym{ids}{IDS}{Intrusion Detection System, Système de Détection d'Intrusion}

\newacronym{ips}{IPS}{Intrusion Prevention System, Système de Prévention d'Intrusion}

\newacronym{one}{ONE}{Open Network Environment, Environnement Réseau Ouvert}

\newacronym{api}{API}{Application Programming Interface, Interface de Programmation}

\newacronym{asic}{ASIC}{Application Specific Integrated Circuit, Circuit intégré pour application spécifique}

\newacronym{iaas}{IaaS}{Infrastructure as a Service, Infrastructure en tant que service}

\newacronym{http}{HTTP}{HyperText Transfer Protcol, Protocole de transfert de hypertexte}

\newacronym{qos}{QoS}{Quality of Service, Qualité de service}

\newacronym{ietf}{IETF}{Internet Engineering Task Force, Détachement d'ingénierie d'internet}

\newacronym{irtf}{IRTF}{Internet Research Task Force, Détachement de recherche d'internet}

\newacronym{aci}{ACI}{Application Centric Infrastructure, Infrastructure centrée sur les applications}

\newacronym{sds}{SDS}{Software Defined-Storage, Stockage Défini par Logiciel}

\newacronym{iass}{IaaS}{Infrastructure as a Service, Infrastructure en tant que Service}

\newacronym{npb}{NPB}{Network Packet Broker}


\newglossaryentry{paradigme}
{
  name=Paradigme,
  text=paradigme,
  description={Un paradigme consiste en une collection de règles, standards et exemples de pratiques scientifiques, partagés par un groupe de scientifiques. 
  Sa genèse et poursuite en tant que tradition de recherche sont conditionnées à un fort engagement et consensus des personnes impliquées. \cite{paradigmdef}
  D'après Dosi \cite{newparadigm}, quand un nouveau paradigme technologique apparaît, il représente une discontinuité ou un changement dans la manière de penser. Ce changement apporté par le paradigme est souvent lié à une sorte d'innovation radicale qui implique une nouvelle technologie. Dans ce document, le terme paradigme sera employé dans ce sens d'innovation et application de nouvelle technologie.  }
}

\newglossaryentry{mpls-glo}
{
  name=MPLS,
  text={MPLS -- Multiprocol Label Switching, Commutation multi-protocoles par étiquettes},
  description={MPLS introduced an explicit distinction between the network edge and the network core. Edge routers inspect the incoming packet headers (which express the host’s requirements as to where to deliver the packet) and then attach a label onto the packet which is used for all forwarding within the core. The label-based forwarding tables in core routers are built not just to deliver packets to the destination, but also to address operator requirements such as VPNs (tunnels) or traffic engineering. MPLS labels have meaning only within the core, and are completely decoupled from the host protocol (e.g., IPv4 or IPv6) used by the host to express its requirement to the network. Thus, the interface for specifying host requirements is still IP, while the interface for packets to identify themselves is an MPLS label. However, MPLS did not formalize the interface by which operators specified their control requirements. Thus, MPLS distinguished between the Host-Network and Packet-Switch interfaces, but did not develop a general Operator-Network interface. }
}



\newglossaryentry{scalability}
{
  name=Scalabilité,
  text=scalabilité,
  description={ Terme provenant de l'anglicisme \textit{scalability} qui exprime la capacité d'être mis à échelle. En informatique cela désigne la capacité d'un système, d'un réseau ou un processus de gérer l'augmentation ou la réduction de la charge de manière à pouvoir la gérer. \cite{scalability}. Le terme est souvent employé pour exprimer une extensibilité, évolutivité ou passage à l'échelle, mais il n'y <<a pas d'équivalent communément admis en français >>. \cite{chevance2001serveurs}  }
}

%n'a pas d'équivalent communément admis en français


\newglossaryentry{abstraction}
{
  name=Abstraction,
  text=abstraction,
  description={  En informatique, l'abstraction est un terme souvent employé pour désigner le mécanisme et la pratique qui réduisent et factorisent les détails négligeables de l'idée exprimée afin de se focaliser sur moins de concepts à la fois.
     C'est aussi la notion de couches d'abstraction utilisée comme moyen pour gérer la complexité des systèmes informatiques où les couches correspondent à des niveaux de détails appliqués. \cite{AbstractionCS}
  }
}


%Scalability is the ability of a system, network, or process to handle a growing amount of work in a capable manner or its ability to be enlarged to accommodate that growth.

\newglossaryentry{bigdata}
{
  name=Big Data,
  description={Big Data est un terme appliqué aux ensembles de données dont la taille (ou le format) est au-delà de la capacité des outils logiciels communs, qui ne peuvent plus les capturer, les gérer et les traiter. Une nouvelle classe de technologies et outils a été développée pour attribuer une valeur commerciale à ces données grâce à une analyse complexe. Le terme est employé en  référence à ce type de données ainsi qu'aux technologies utilisées pour les stocker et les traiter.
  \cite{IMBigData}    }
}


\newglossaryentry{cluster}
{
  name=Cluster,
  text=cluster,
  description={  En réseaux informatiques, un cluster désigne un groupe des machines reliées entre elles à l'aide d'un réseau de communication. Cette configuration est souvent utilisée pour réaliser des calculs à haute performance. \cite{cluster} }
}


\newglossaryentry{cloudcomputing}
{
  name=Cloud Computing,
  text=cloud computing,
  description={  Cloud Computing, ou informatique dans les nuages, est une évolution de la fourniture de services \gls{ti} qui offre un moyen d'optimiser l'usage et le déploiement rapide de ressources. Cela se fait par des systèmes et solutions plus efficaces et \glslink{scalability}{scalables}, fournissant un niveau plus haut d'automatisation. Diverses entreprises ont adopté le cloud computing et réalisent des avantages significatifs en agilité, réduction de coûts et soutien de la croissance du business. \cite{CloudComputingIntelVisionSpeeding}      }
}



\newglossaryentry{virtualisation}
{
  name=Virtualisation,
  text=virtualisation,
  description={  Pour diverses entreprises, l'infrastructure serveur virtualisée est la base sur laquelle le \glslink{cloudcomputing}{cloud} est construit. Initialement, les technologies de virtualisation ont permis aux data centers de consolider leurs infrastructures pour réduire les coûts. Avec le temps, l'intégration des technologies pour le management flexible de ressources a facilité une allocation plus dynamique. Cela a aidé à réduire les coûts et a également augmenté la flexibilité et la performance. \cite{CloudComputingIntelVisionSpeeding}  }
}



\newglossaryentry{datacenter}
{
  name=Data  Center,
  text=data center,
  description={  Centre de traitement de données. Il s'agit d'une installation utilisée pour héberger des systèmes informatiques et les composants associés, comme les systèmes de télécommunication et de stockage. En général, un data center inclut alimentation et  connexions des données redondantes, contrôles d'environnements comme la climatisation ainsi que divers dispositifs de sécurité. \cite{dataCenterDef} }
}
\newglossaryentry{middlebox}
{
  name=Middlebox,
  text=middlebox,
  description={  Boîtier intermédiaire. Un middlebox est un serveur conservant des états de la communication entre deux hôtes. Ils se différencient des hôtes qui représentent les extrémités de la communication. Ils sont encore différents des routeurs qui ne gardent pas d'états concernant les sessions de communications.  \cite{InternetEvolutionRoleSoftwareEngineeringRealInternet}  },
  plural=middleboxes
}

\newglossaryentry{openflow}
{
  name=OpenFlow,
  description={   Le protocole OpenFlow vise à standardiser l'interface entre les applications et le contrôleur  ainsi que l'interface entre le contrôleur et les éléments de commutation. \cite{SurveySDNArchi} \cite{OpenFlowStanfordSwitch}  }
}


\newglossaryentry{controlplane}
{
  name=Plan de Contrôle,
  text=plan de contrôle,
  description={  Intelligence du réseau, ensemble des données locales utilisées pour établir les entrées des tableaux de commutation, qui sont utilisés par le plan de données pour effectuer la transmission du trafic entre les ports d'entrée et de sortie du dispositif. \cite{sdnbookControlDataPlanes} },
  plural={Plans de contrôles}
}

\newglossaryentry{dataplane}
{
  name=Plan de Données,
  text=plan de données,
  description={  Le plan de donnés traite les data-grammes entrants dans le média à travers une série d'opérations au niveau des liens qui collectent ces data-grammes et réalisent divers tests de cohérence basiques. Ensuite les data-grammes sont transférés en accord avec des tableaux pré-remplis par le \gls{controlplane}.  \cite{sdnbookControlDataPlanes}},
  plural={plans de données}
}

\newglossaryentry{opensource}
{
  name=Open Source,
  text=open source,
  description={ Logiciel avec code source ouvert, qui peut donc être utilisé librement, modifié et partagé par quelqu'un. Un logiciel open source est développé par plusieurs personnes et distribué sous des licences qui se conforment à la définition d'open source.  \cite{OpenSource}  }
}

\newglossaryentry{opendaylight}
{
  name=Open Daylight,
  description={ Association initiée par Linux Foundation pour l'union des géants du marché réseau dans le but de développer un contrôleur SDN open source, pour l'innover, l'encourager et pour permettre son adoption accélérée. \cite{OpenDaylight} }
}


\newglossaryentry{fabric}
{
  name=Fabric,
  text=fabric,
  description={ En informatique, fabric (qui signifie tissu en anglais) est un synonyme de plate-forme ou structure.  En général, le terme fabric décrit la façon dont différents composants travaillent ensemble pour former une entité unique. Dans ces systèmes la liaison entre les composants est tellement dense qu'un schéma représentant leurs relations rassemblerait à une pièce de tissu tricotée.
  Sous ce terme généralement admis par l'industrie réseau, un fabric est une topologie réseau dans laquelle les composants transmettent des données l'un à l'autre  à travers les switches d’interconnexion.
  % En théorie, un \textit{fabric} devrait être capable de supporter un grand nombre de conceptions de pointe y compris des schémas d'adressage et des modèles de politique. 
  \cite{NetworkFabricSearchSDN} \cite{fabricExtending}}
}

\usepackage[english,francais]{babel}
\frenchbsetup{og=«, fg=»}



\pagestyle{scrheadings}

\begin{document}
\includepdf[pages={1-2}]{couverture-ebt.pdf}

\frontmatter
%\begin{flushright}
%It is a sad age when it is more difficult to break a prejudice than an atom.\\
%Albert \bsc{Einstein}\\
%\end{flushright}

\section*{Remerciements}

Le présent document peut être considéré comme le résumé de mon cycle d'études \og Master Réseaux Informatique d'Entreprise \fg{}.

Je tiens à  remercier tous ceux qui, à l'Institut National Polytechnique de Grenoble et à l'entreprise Bull, m'ont accompagnée dans ce parcours. 

À tous les intervenants pédagogiques de la formation RIE, j'adresse mes remerciements pour les connaissances transmises,  spécialement à Messieurs Kobylanski et Parouty ainsi qu'à Madame Panne, pour leurs conseils et leur accompagnement.

J'exprime toute ma gratitude à mes collègues de Bull pour leur accueil et convivialité pendant ces deux ans d'alternance ; Spécialement aux membres de l'équipe OSI pour m'avoir intégrée, je témoigne sincèrement de ma reconnaissance : tout d'abord à Claude Casery, mon tuteur, pour son rôle clé dans mon développement professionnel, mais aussi à Julien et à Laure pour leur aide technique et leur disponibilité.

Merci également aux membres de l'équipe OMNIS, qui m'ont précédemment accueillie, pendant mon stage de licence, et qui sont devenus rapidement mes grands amis de France. Leur sagesse, encouragements (notamment de la part de Martine) et drôlerie m'ont beaucoup inspirée et ont contribué à l'aboutissement de cette expérience à l'étranger pour moi. 

Dans l'équipe Storeway, je suis particulièrement reconnaissante envers Emmanuel de m'avoir initiée au sujet SDN, thème choisi pour ce mémoire, ainsi qu'à Renata, ma compatriote, pour sa compréhension et son aide pour l'équivalence d'un vocabulaire non évident. 

J'exprime ma gratitude aussi à tous ceux qui ont relu mon document et m'ont permis de l'améliorer, mais particulièrement à Christine pour les corrections, patience et explications.

Enfin, je remercie ma famille et mes amis qui m'ont soutenue même de loin. Dans cette ouverture, un merci aussi à ceux pour lesquels j'ai cultivé une forte affection depuis mon arrivée. Tous ont contribué d'une manière ou de l'autre à la réussite de ce parcours.


\tableofcontents
\listoftables
\listoffigures

\mainmatter
%\addchap{Introduction}
\addchap{Introduction}

%Most IT infrastructures were not built to support the explosive growth in computing capacity and information that we see today. Many data centers have become highly distributed and somewhat fragmented. As a result, they are limited in their ability to change quickly and support the integration of new types of technologies or to easily scale to power the business as needed.

%When equipped with a highly efficient, shared, and dynamic infrastructure, along with the tools needed to free up resources from traditional operational demands, IT can more efficiently respond to new business needs. As a result, organizations can focus on innovation and on aligning resources to broader strategic priorities. Decisions can be based on real-time information.

Les centres de traitement de données\index{Data Centre} évoluent aujourd'hui à un rythme intense pour accompagner l'explosion constatée dans l'utilisation (en volume et en diversité) de données. L'accélération de l'innovation\index{Innovation} dans l'informatique impose une rénovation constante des infrastructures des entreprises. La virtualisation\index{Virtualisation} a permis aux centres de données d'améliorer la productivité de ses serveurs, mais pour arriver à l'agilité\index{Agilité} souhaitée, les data centres doivent faire évoluer leurs réseaux. % pour pouvoir passer au Cloud Computing. 
Cette étude analyse les applications \gls{sdn}\index{SDN} pour distinguer quels sont les apports dans le contexte actuel et futur des data centres et habiliter le passage au \gls{cloudcomputing}.\index{Cloud Computing}

\par 
La plupart des infrastructures de \gls{ti} n'ont pas été construites pour supporter la croissance explosive des données ni la capacité de traitement de l'information\index{Compute} observée aujourd'hui. Plusieurs centres de traitement de données sont devenus hautement distribués et relativement fragmentés par rapport aux besoins des différents profils des clients. Ils se trouvent donc limités dans leur capacité à évoluer rapidement et à supporter l'intégration des nouveaux types de technologies ou à se mettre à l'échelle des besoins de ses utilisateurs.

\par 
Lorsqu'ils sont équipés d'infrastructures performantes, partagées et dynamiques ainsi que des outils nécessaires pour adapter les ressources à la demande, les \gls{si} peuvent répondre efficacement aux besoins métiers. Ainsi, les structures pourraient se focaliser sur l'innovation\index{Innovation} l'adaptation des ressources selon leurs priorités stratégiques métiers. Cela faciliterait la prise de décisions, qui pourrait être concentrée sur l'information en temps-réel. \cite{hpAlcatelCreatinCloudDCchallenges}

\par
Alors que le coût du réseau dans un data centre est estimé à 15\% du total, sans être un des plus élevés, il est largement établi qu'il représente un élément clé pour la réduction des coûts\index{Coûts} et l'augmentation du retour sur investissement. Les coûts d'investissement dans les serveurs ont été évalués à 45\% des coûts des data centres. Malheureusement la charge\index{Charge} utile des serveurs est remarquablement basse, arrivant à seulement 10\% d'utilisation dans certains exemples.  \cite{cloudCosts}

\par 
La technique de la virtualisation\index{Virtualisation} a permis le partage des processus entre machines, mais des contraintes réseau continuent à limiter l'agilité\index{Agilité} dans les data centres. L'agilité est définie par la capacité d'affecter tout service n'importe où dans le data centre, tout en assurant la sécurité\index{Sécurité}, la performance et l'isolation entre tous les services. Les designs des réseaux conventionnels dans un data centre empêchent cette agilité ; par nature ils fragmentent  à la fois les réseaux et la capacité des serveurs, limitant et réduisant la croissance dynamique des pools de serveur et de traitement de l'information\index{Compute}. \cite{cloudCostsAgility}



\par 
L'agilité est donc un élément clé ; certaines entreprises s'évertuent à déployer des nouvelles applications ou faire évoluer les existantes au rythme de la croissance de leur business. Selon le sondage mené par AlgoSec avec 240 professionnels de l'informatique, 25\% des organisations participantes doivent attendre plus de 11 semaines pour qu'une nouvelle application soit mise en ligne (et dans 14\%, ce temps dépasse 5 mois). Les résultats révèlent également que 59\% des entreprises nécessitent plus de huit heures pour réaliser un changement de connectivité dans une application. \cite{algoSecSurvey}


\par
Cependant, lors du passage au Cloud, les entreprises réalisent que la virtualisation\index{Virtualisation} des serveurs est considérablement limitée par les designs Ethernet classiques et les contrôles de sécurité réseau traditionnels. Avec l'augmentation de la virtualisation au sein des data centres, quatre tâches majeurs deviennent critiques :
\begin{itemize}
\item Prévention de la congestion du trafic ;
\item Réduction de la complexité\index{Complexité} mise en place des politiques réseau et maintien du niveau de service;
\item Élimination des points aveugles qui conduisent à des pannes ;
\item  Scellement des failles de sécurité pour protéger les données. \cite{virtualizedCCCC}\\
\end{itemize}
%But as businesses move to the private cloud, they are finding server virtualization is severely limited by clas- sic Ethernet designs and traditional network security controls. As data center virtualization scales, four critical tasks become increasingly cumbersome:n Preventing traffic bottlenecksn Reducing complexity of network policy and service level assurancen Eliminating management blind spots that lead to outagesn Sealing up security loopholes to protect data

\par
Cette étude a pour but de démontrer comment SDN peut être appliqué aux data centres pour permettre le Cloud Computing et en dépasser ces limites du réseau typique actuel. Dans le premier chapitre, le contexte des data centres sera défini. Ensuite, les problématiques dans l'aspect réseau seront exposées. Enfin, le dernier chapitre présentera SDN et démontrera ses apports dans ce cadre.




\chapter{Les évolutions du business modèle des Data Centres}
\label{chap-1}

Ce chapitre a pour but de définir un data centre afin de pouvoir analyser ses problématiques, enjeux et solutions possibles, en vue de comprendre son état actuel et ses limites par rapport aux nouveaux besoins et défis business. Les éléments les plus importants de la conception et de l'architecture du data centre seront présentés ainsi que les difficultés qui l'ont amené faire à évoluer son modèle de livraison vers l'approche Cloud Computing\index{Cloud Computing}.

\section{Data centres et leurs objectifs}

Un data centre (aussi nommé \og ferme de serveurs \fg{}) est un répertoire centralisé pour le stockage\index{Stockage}, le management et la distribution de données et d'informations. Typiquement, un data centre\index{Data Centre} est une installation utilisée pour loger des systèmes informatiques et ses composants associés, tels que systèmes de télécommunication et stockage. 

Les data centres traditionnels hébergent historiquement de nombreuses applications relativement petites ou moyennes, chacune s'exécutant dans une infrastructure matérielle dédiée qui est isolée et protégée des autres systèmes dans la même installation. Ces data centres accueillent du matériel et du logiciel pour multiples unités organisationnelles ou même diverses entreprises. Différents systèmes informatiques au sein d'un tel data centre ont souvent très peu d'éléments en commun en termes de matériel, logiciel ou infrastructure de maintenance, et en général ne communiquent pas entre eux. 


Les tendances de l'informatique vers une approche côté serveur et l'explosion en popularité des services sur internet ont changé ce scénario. Des infrastructures data centre entières ont été dédiées à un seul acteur pour faire fonctionner ses services offerts. Dans ce cadre, un data centre appartient à une seule organisation et utilise des matériels et plateformes logicielles relativement homogènes qui partagent une couche commune de systèmes de management. Ces data centres dédiés exécutent un nombre réduit d'applications (ou services internet) beaucoup plus importants en taille; l'infrastructure commune de management permet alors une meilleure flexibilité de déploiement. 

Ces infrastructures\index{Infrastructure} sont montées pour gérer la taille des applications\index{Application} déployées et la haute disponibilité exigée pour ces services, visant en général 99,99\% de durée de fonctionnement (une heure au maximum de temps d'arrêt par an). Il est difficile d'atteindre un fonctionnement libre des failles dans un regroupement de systèmes matériel et cela devient encore plus complexe avec le grand nombre de serveurs impliqués.

\par
Les infrastructures de ces data centres doivent être dimensionnées précisément  en fonction de la charge des applications supportées. Par conséquent, des nouvelles approches ont été proposées pour la construction et l'opération de ces systèmes qui doivent être conçus pour tolérer un nombre important de failles avec très peu ou aucun impact sur la performance et disponibilité des services offerts. \cite{understandingCloudWhatDC}  \cite{datacenterAsComputerIntro}

\section{Organisation d'un data centre et difficultés}

%A data center is generally organized in rows of ‘‘racks” where each rack contains modular assets such as servers, switches, storage ‘‘bricks”, or specialized appliances
Un data centre est en général organisé en lignes de racks où chaque rack\index{Rack} contient des dispositifs modulaires tels que serveurs, switches, briques de stockages ou instruments spécialisés. %Trois principaux éléments d'infrastructure constituent les data centres : le stockage, le réseau et l'approvisionnement énergétique.
Des composants essentiels de l'infrastructure, branchés aux racks des data centres d'entreprises tels que compute, stockage et réseau, sont la base sur laquelle les applications business sont construites. Un châssis se présente avec ses propres ventilateurs, source d'alimentation, panier d'interconnexion et module de management. 
Pour réduire l'espace occupé, des serveurs peuvent être compartimentés dans un châssis qui est glissé dans le rack. Un châssis fournit des slots de taille standard où il est possible d'insérer des élément actifs modulaires (aussi connus en tant que \og blades \fg{}). Un seul châssis peut contenir 16 serveurs 1 U; étant donné que les racks supportent 6 châssis, ils ont une capacité théorique de 96 éléments modulaires.


\begin{figure}[h]
\begin{center}
\includegraphics[width=0.7\textwidth]{images/racks} 
\caption{Organisation de racks. \cite{datacenterAsComputerIntro}}\label{racks}
\end{center}
\end{figure}

La figure \ref{racks} montre l'organisation des racks dans un data centre. Un serveur occupant 1 U du rack est visualisé à gauche. Au centre on peut voir un rack et à droite un cluster de racks avec un swtich/routeur\index{Routeur} de niveau cluster. En général un ensemble de serveurs 1U sont montés dans un rack et inter-connectés à un commutateur Ethernet local. Ces switches\index{Commutateur!switch} au niveau des racks, qui peuvent utiliser des liens de 1 à 10 Gbps, ont un nombre de connexions montantes vers un ou plus switches de niveau cluster (data centre).

Le stockage dans les data centres peut être proposé de diverses manières. Souvent le stockage de haute performance est logé dans des \og  tours de stockage \fg{} qui permettent un accès distant transparent au stockage, indépendamment du nombre et des types des dispositifs de stockage physiques installés. Le stockage peut également être fourni dans une \og  brique de stockage \fg{} plus petite, localisée dans le rack ou slot de châssis ainsi que directement intégrée aux serveurs. Dans tous les cas, un accès réseau efficace au stockage est crucial.

Le problème le plus important dans cette structure est l'éventuelle insuffisance de bande passante. En général, les connexions montantes sont conçues pour supporter un certain taux de demandes excédentaires puisque la fourniture d'une bande passante entière n'est pas toujours possible. Par exemple, pour 20 serveurs à 1Gbps, chacun doit partager un lien Ethernet montant unique de 10Gps à un taux de demande excédentaire de 2. Cette situation peut être problématique si la charge réseau non locale monte considérablement. Comme le stockage est traditionnellement fourni dans une tour séparée, tout le trafic\index{Trafic} de stockage traverse le lien montant dans le réseau stockage. Par exemple, l'archivage d'un gros volume peut consommer une importante bande passante. À mesure que les data centres augmentent en taille, une architecture réseau plus extensible\index{Extensible} devient essentielle.

La consommation d'énergie est également une des préoccupations de la conception des data centres, car les coûts\index{Coûts} liés sont devenus une part importante de la totalité des coûts pour cette classe de systèmes. Actuellement les CPUs ne sont plus le seul élément cible d'amélioration de l'efficacité\index{Efficacité} énergétique\index{Énergie!énergétique}, car ils ne dominent plus la majorité de la consommation. Des problématiques de ventilation et surconsommation d'énergie sont des facteurs de plus en plus critiques dans la conception de data centres.\cite{datacenterAsComputerIntro} \cite{dataCenterEvolution}

\section{Virtualisation et partage de ressources}

Le besoin d'augmenter l'efficacité dans l'utilisation des ressources a conduit à une conception d'infrastructures avec partage de ressources et virtualisation. La virtualisation\index{Virtualisation} fait référence à l'abstraction des ressources logiques de leurs couches physiques pour améliorer l'agilité, la flexibilité et la réduction des coûts et ainsi privilégier le business. La virtualisation permet de consolider un ensemble de composants d'infrastructures sous-utilisés en un nombre de dispositifs plus petits et mieux utilisés, contribuant à l'économie des coûts.

La virtualisation de serveurs est une méthode pour abstraire le système d'exploitation de la plateforme matérielle. Cela permet aux multiples systèmes d'exploitation ou multiples instances du même système d'exploitation de coexister dans un ou plusieurs processeurs. L'image \ref{virtinfra} illustre le partage de ressources\index{Ressources} par l'intermédiaire de la virtualisation.

\begin{figure}[h]
\begin{center}
\includegraphics[width=0.98\textwidth]{images/shared_infa_virt} 
\caption{Modèle d'infrastructure à ressources partagées. \cite{journeySDDC}}\label{virtinfra}
\end{center}
\end{figure}

Un hyperviseur\index{Hyperviseur} ou moniteur de machines virtuelles\index{VM!machines virtuelles} est inséré entre le système d'exploitation et le matériel pour réaliser la séparation entre le logique et le physique. Les instances de systèmes d'exploitation lancées sont appelées invités, ou systèmes d'exploitation invités. L'hyperviseur fournit l'émulation matérielle aux systèmes invités et gèrent l'allocation de ressources matérielles.  Les principaux hyperviseurs disponibles sur marché aujourd'hui sont : VMware\index{VMware} ESXi, KVM basé sur Linux et supporté par Red Hat, Citrix XEN et Microsoft Hyper-V. 

%Ce modèle apporte des avantages en termes de comment les ressources sont efficacement utilisées avec des charges applicatives idéales.
Ce modèle apporte des avantages pour l'efficacité\index{Efficacité} dans l'utilisation de ressources avec des charges applicatives idéales.
 Cependant, quand une application commence à consommer plus de ressources que l'estimé, il peut arriver des scénarios où les systèmes d'exploitation invités n'ont pas assez de ressources, impactant ainsi la qualité du service business offert. 

Cette approche a apporté une maitrise globale de management, monitoring et outillage. Elle a aussi mis en évidence que le composant \og compute\index{Compute} \fg{} de l'infrastructure améliore clairement l'utilisation et automatisation des ressources serveurs. Cette amélioration a été possible grâce à la programmation du contrôle de ressources fournies aux instances invitées. Toutefois, le développement de nouvelles solutions pour gérer la charge dynamique de certaines application faisait toujours défaut. \cite{journeySDDC} \cite{ibmPlanningVirtCCchap2}

%This model has its advantages in terms of how resources are efficiently utilized in ideal applica- tion workloads. However, when one or more appli- cation workloads begin to consume more resourc- es than expected, scenarios could arise where several guest operating systems are short of compute resources, thereby impacting business application service level agreements.
%While this approach brought holistic capacity management, monitoring and tools capabilities, it also provided evidence that infrastructure compute and server resources were truly ben- efiting from improved resource utilization and automation. This was brought about, to a certain extent, by programmatically controlling the resources provided to guest instances. However, new thinking about solutions was still needed to meet the challenges of dynamic workloads of run- the-business applications and compute-intensive enterprise applications.



\section{Le besoin d'un modèle plus dynamique}


Traditionnellement, les data centres\index{Data Centre} d'entreprises sont conçus pour durer pour toujours et atteindre les objectives déterminés de l'économie. Cela veut dire que les éléments sous-jacents sont dimensionnés\index{Dimensionnement} et construits pour supporter le pic de charge\index{Charge} projeté en termes de performance, disponibilité et sécurité. Quand la croissance volumétrique projetée ne correspond pas à la réalité, cette méthode de dimensionnement peut conduire à une situation de sous-dimensionnement ou sur-dimensionnement. Ce qui apporte un effet négatif pour les investissements et les effort de réduction de coûts\index{Coûts}.

En général, pour atteindre une meilleure disponibilité, les infrastructures sont amenées à une sous-utilisation des ressources. Comme la charge des applications varient continuellement dans les applications sur internet, il reste deux choix : soit sous-dimensionner la provision et perdre des clients ou alors sur-dimensionner et gaspiller les ressources. 

Dans tous les cas, un plan détaillé de capacité est fait pour spécifier une série d'investissements importants en matériel et logiciel, dont la charge maximale est déterminée. L'image suivante illustre cette planification et les situations de problèmes de dimensionnement.


\begin{figure}[h]
\begin{center}
\includegraphics[width=0.98\textwidth]{images/fixed_capacity_load_prediction} 
\caption{Capacité fixe de ressources vs charge prévisionnelle. \cite{awsScaling}}
\end{center}
\end{figure}

Face à cette problématique, un nouveau mode de livraison\index{Delivery Model!Mode de livraison} a été proposé pour aborder les défis du traitement des demandes pour la variation dynamique des charges applicatives. Avec la nouvelle tendance du Cloud Computing et l'\gls{iaas}\index{IaaS}, la conception de \glspl{cluster}\index{Cluster} hautement disponibles et des solutions extensibles peut être architecturée avec des requis non-fonctionnels comme base. 
%With the emergence of the cloud, the new age “mantra” and infrastructure as a service (IaaS) as a delivery model (as illustrated in Figure 3), the challenges of processing demands from dynamic workloads is being addressed. Designing high- availability clusters and scalable solutions can be architected based on nonfunctional requirements.

Avec sa nature extensible, le modèle de livraison cloud permet aux ressources\index{Ressources} d'être étendues\index{Extensible!étendues} et réduites  dynamiquement en fonction de la consommation. Une couche logicielle d'abstraction, implémentée par les hyperviseurs, virtualise le traitement des ressources physiques,  permettant ainsi au processeur\index{Compute!processeurs}, à la mémoire et aux disques durs de s'accommoder aux variations des demandes. \cite{journeySDDC} \cite{awsScaling}

%One of the biggest technical challenges of running an online business is how well they are able to handle the scalability requirements.  The Load traffic pattern keeps varying for online businesses and accordingly they will have to scale and maintain the acceptable performance levels.  Since the Traffic patterns are fluctuating in online business, they either tend to under provision and loose customers (or) over provision and waste hardware + costs. This problem is well illustrated in the below diagrams.
%Business usually makes detailed capacity planning and large upfront investment in their hardware and software. This HW/SW’s are usually provisioned with fixed capacity.

%Traditionally, enter-prise data centers are designed to last forever and meet visible business objectives, meaning that their underlying components are sized and built for a projected workload. They are also sized and built using application volumetric modeling and nonfunctional requirements such as perfor-mance, availability, scalability and security.

%The infrastructure is designed and provisioned considering the specific volumetric for support- ing the business applications and considering the peak load transaction in jobs per second, avail- ability and scalability requirements. When volu- metric and projected growth do not manifest as envisaged, this method of sizing infrastructure compute and storage could lead to either under- sizing or oversizing the footprint. Often, having such islands of infrastructure compute and storage leads to underutilization of resources. This has a cascading effect on investment and the effort expended toward energy consumption, management overheads, software licenses and data center costs.


%The shortcomings of this model led many enter- prises to the next wave of infrastructure design — utilizing shared infrastructure services and virtualized compute to increase efficiency in resource utilization and ensure that infrastruc- ture is designed and fit for the purpose, and not over-engineered.

%Virtualization refers to the abstraction of logical resources away from their underlying physical resources to improve agility and flexibility, reduce costs, and thus enhance business value. Virtualization allows a set of underutilized physical infrastructure components to be consolidated into a smaller number of better utilized devices, contributing to significant cost savings.

%Server virtualization is a method of abstracting the operating system from the hardware platform. This allows multiple operating systems or multiple instances of the same operating system to coexist on one or more processors. A hypervisor or virtual machine monitor (VMM) is inserted between the operating system and the hardware to achieve this separation. These operating systems are called “guests” or “guest OSs.” The hypervisor provides hardware emulation to the guest operating systems. It also manages allocation of hardware resources between operating systems.

%According to a recent Gartner study, the leading challenges facing today’s data centers are intrinsic to many of the aforementioned business drivers and their associated IT solutions. Top challenges cited include: 
%• Keeping up with data growth 
%• Maintaining system performance and scalability
%• Mitigating network congestion and connectivity issues
%• Minimizing power, cooling and space costs
%• Effectively managing the data center and its infrastructure
%Furthermore, according to a 2011 survey of the Data Center Users’ Group (DCUG), the leading infrastructure challenges included data center availability, high heat densities, energy efficiency and maintaining adequate power densities (see Figure 1). Each of these challenges resonates closely with the leading data center challenges faced by IT professionals.


%\section{Virtualisation}



%Integrated infrastructure solutions are specifically designed to provide advantages compared to a conventional physical infrastructure because they are: 
%•	 Efficient in power usage, space utilization and IT employee productivity
%•	 Economical in initial cost by making use of existing infrastructure and not requiring expensive room upgrades
%•	 Interoperable through simplified design and implementation of systems and components 
%•	 Controllable through planning, monitoring and management over the changing IT environment



%\section{Tendances des meilleures pratiques}


%Integrated infrastructure solutions are specifically designed to provide advantages compared to a conventional physical infrastructure because they are: 
%•	 Efficient in power usage, space utilization and IT employee productivity
%•	 Economical in initial cost by making use of existing infrastructure and not requiring expensive room upgrades
%•	 Interoperable through simplified design and implementation of systems and components 
%•	 Controllable through planning, monitoring and management over the changing IT environment
%\section{Architecture}

%\chapter{Cloud Computing}

% Un regard sur le nouveau \og business model \fg{} apporté par le Cloud Computing, les bénéfices de son adoption et les enjeux pour les infrastructures qui doivent répondre à ce nouveau paradigme, est l'objet de ce chapitre. Il sera démontré pour quelles raisons il nécessaires de faire évoluer les infrastructures actuelles vers le Cloud et pourquoi il n'est pas encore largement adopté.

\section{Cloud Computing}
En termes très simples, le Cloud Computing\index{Cloud Computing} peut être défini comme un nouveau modèle de consommation et livraison\index{Delivery Model} de ressources de \gls{ti} et de services métiers, et est principalement caractérisé par :
\begin{itemize}
\item Libre service à la demande;
\item Service réseau très accessible;
\item Location indépendante de services en commun;
\item Extensibilité et approvisionnement rapides;
\item Paiement à la consommation.
\end{itemize}
%In very simple terms, cloud computing is a new consumption and delivery model for information technology (IT) and business services and is characterized by:
% • On-demand self-service
% • Ubiquitous network access
% • Location-independent resource pooling
% • Rapid elasticity and provisioning
% • Pay-per-use

\par
Les avancements importants dans la virtualisation\index{Virtualisation}, réseau\index{Réseau}, approvisionnement et architectures multi-tenantes\index{Multi-Tenant} ont permis de faire évoluer radicalement les infrastructures de data centres. Le plus grand impact du Cloud Computing\index{Cloud Computing} vient de l'instauration de nouveaux modèles de consommation et de livraison de services qui supportent l'innovation\index{Innovation} du business\index{Business}.

%Cloud has evolved from on demand and grid computing, while building on significant advances in virtualization, networking, provisioning, and multitenant architectures. As with any new technology, the exciting impact comes from enabling new service consumption and delivery models that support business model innovation.


L'évolution des data centres a permis de rendre service à une plus grande variété de besoins dans le monde du travail, ce qui implique la prise en compte de plusieurs facteurs lors de la conception face à différents objectifs. Le Cloud Computing est donc né en tant que nouveau paradigme pour les architectures data centre.

%As we have seen, data centers have grown to serve a wide range of business needs, and there are many factors to consider when designing a solution that meets different objectives. Within the past several years, a powerful new paradigm has emerged that has important implications for data center architectures and how they meet these varied objectives. This is the paradigm of cloud computing.

Le Cloud Computing livre dynamiquement des services sur des réseaux\index{Réseau} à partir d'un ensemble abstrait de ressources. Ces ressources se retrouvent quelque part dans le \og nuage \fg{} (symbole qui fait allusion à la représentation d'internet dans les topologies réseau) disponibles immédiatement à la demande. Les types de ressources ainsi que leur localisation sont transparents aux utilisateurs finaux. Ces utilisateurs se soucient principalement que leurs applications, données et contenus soient sécurisés et disponibles, avec un niveau de qualité spécifié.

%Cloud computing delivers services dynamically over networks from an abstracted set of resources. The resources are somewhere in the cloud and available on demand. The types of resources and their location are transparent to end users. End users primarily care that their applications, data and content are secure and available, with a desired level of quality.

Du point de vue de l'infrastructure, le Cloud Computing fait des fortes demandes aux ressources\index{Ressources} mutualisées dans une variété de technologies (de compute, de stockage, de réseau) pour leur allocation dynamique. Tout ceci dans un environnement automatisé, orchestré et logiquement diversifié, en conciliant une variété d'applications. L'orchestration\index{Orchestration} permet de mutualiser les ressources\index{Ressources} à travers multiples data centres pour une réponse dynamique aux besoins clients. 


La virtualisation de serveurs a représenté un premier et important pas pour la viabilité de l'approche Cloud Computing. Toutefois, les autres deux éléments de base de l'infrastructure data centre doivent accompagner ces changements pour autoriser un accès complet aux services offerts par le Cloud. Plus spécifiquement la couche d'abstraction, assurée par les hyperviseurs\index{Hyperviseur} qui ont permis la séparation des systèmes logiques des serveurs physiques dans le cas de la virtualisation de serveurs\index{Compute!Virtualisation des serveurs}, doit être également appliquée aux matériels réseaux\index{Virtualisation du Réseau} et de stockage\index{Virtualisation du Stockage}. Cela permettra la définition d'un data centre entièrement piloté par du logiciel\index{Software-Defined Data Center!Data centre piloté par logiciel} qui gère\index{Gestion} les ressources physiques, en les activant selon la charge applicative\index{Charge} spécifiée à assurer.

%\pagebreak

 L'image suivante illustre une vue conceptuelle d'un data centre basé sur ces trois éléments avec des couches d'abstraction permettant de sécuriser, orchestrer et livrer ses ressources aux consommateurs.

\begin{figure}[h]
\begin{center}

%\begin{wrapfigure}{r}{0.6\textwidth}
\includegraphics[width=0.85\textwidth]{images/CloudRefArchi} 
\caption{Vue conceptuelle d'un data centre. \cite{ciscoCCDCStrategyArchiSolutions}} \label{cloud_conceptual_view}
%\end{wrapfigure} 

\end{center}
\end{figure}

L'abstraction\index{Abstraction} de ces trois composants matériels est essentielle pour achever le mode de livraison Cloud au sein des data centres. L'adoption de la virtualisation des serveurs a déjà atteint son grand public ; en 2009 un sondage avait révélé que 77\% des répondants déployaient au moins un système virtualisé dans leur data centre \cite{x86ServersVirtualization}. On observe qu'actuellement beaucoup des travaux en recherche et développement se déroulent pour acquérir un niveau équivalent de maturité pour les dispositifs réseau et stockage. 

L'abstraction du stockage signifie la capacité à mutualiser les dispositifs physiques de stockage pour pouvoir les utiliser en tant que volumes de stockage logiques. C'est qui caractérise la virtualisation du stockage ou le \gls{sds}.\index{Virtualisation du stockage!\gls{sds}} Pour l'aspect stockage, il est reconnu que des solutions \gls{sds} se trouvent disponibles sur le marché, telles que EMC ViPR, HP\index{HP} StoreVirtual, IBM SmartCloud Virtual Storage Center entre autres.

De manière similaire, il se développe pour les réseaux une technologie fournissant une couche d'abstraction pour divers dispositifs réseau afin de permettre l'isolation logique et l'indépendance du matériel. Il se trouve que \gls{sdn}\index{SDN} est une des approches proposées pour traiter la problématique de l'abstraction réseau et fait donc l'objet de cette étude. Le chapitre suivant démontrera en quoi les réseaux traditionnels ne sont pas adaptés aux exigences du Cloud Computing et analysera des exemples sur divers problèmes rencontrés. L'approche SDN et ses apports seront détaillés par la suite.
\cite{ibmPlanningVirtCCchap1}  \cite{cloudReadyJuniperReferenceDef} \cite{journeySDDC} \cite{ciscoCCDCStrategyArchiSolutions}


\chapter{Problématiques réseau rencontrées}
Dans ce chapitre, les principales problématiques data centre dans un aspect réseau seront présentées et analysées.

\section{Différents usages}

\section{Agilité}

\section{Sécurité}
%\chapter{Enjeux de SDN}

Ce chapitre présente quels sont les enjeux pour déployer SDN. Il a pour but d'identifier les challenges lors de la mise en place de cette architecture ainsi que les problèmes susceptibles d'être rencontrés. Pour chaque enjeux, les idées pour les surmonter sont montrées tout en proposant les compromis de ces solutions.

\section{Contrôle : centralisé ou distribué}
Contrôle centralisé = un seul point de faille pour le réseau complet.

Architecture physiquement distribuée mais centralisé au niveau logique.

Consistence et stateliness quand on distribue des états sur le réseau peut causer un mal comportement des applications qui pensent qu'elles une vision précise du réseau.

\section{Niveau de granularité}

\section{Politiques : réactives ou pro-actives}

\section{Fonctions de Virtualisation du Réseau}
%\gls{nfv}

\chapter{Applications SDN}

Ce chapitre redéfinira SDN et présentera ses réponses aux problématiques réseau rencontrées en général dans les data centre, discutées dans le chapitre précédent.

\section{Redéfinition de SDN}

\section{Solutions}
%\chapter{Apports de SDN aux data centres}
%Ce chapitre démontre les apports de SDN au sein des data centres par rapports aux problématiques présentées précédemment.

\section{Scénarios d'utilisation}

Un système cloud qui s'intègre de façon transparente et dynamique avec un réseau programmable (grâce à SDN) peut fournir une importante plus-value à ses opérateurs et à leurs abonnés (consommateurs finaux et entreprises). Aujourd'hui la connectivité seule ne suffit pas, les utilisateurs réclament une variété de services hébergés dans le cloud, et cela exige des réseaux la capacité de fournir la connectivité correcte à l'application souhaitée. C'est dans ce cadre que  la réelle valeur d'un cloud à réseau dynamiquement programmable  devient visible.
%A cloud system that integrates seamlessly with a real-time, programmable network – enabled by Service Provider SDN – can provide significant value to network operators and their subscribers (both consumers and enterprises). Today, most subscribers do not rely on connectivity alone. Instead, they demand a wide range of services that are cloud-hosted, and they require the network to play a role in offering the right connectivity for the desired application. This is where the real value of a Service Provider SDN-based, real-time programmable network and cloud becomes apparent.

Cette capacité permet de découper le réseau en tranches et offrir aux clients leurs morceaux dédiées et personnalisés. Il existe une variété de scénarios imaginables  à partir de ce concept de diviser le réseau pour convenir à différents applications et besoins.
%A “meta” use case is the ability to slice and offer consumers/enterprises a piece of the network-plus-cloud for their dedicated, personalized use. There are multiple variants of use cases that are based on this concept of the ability to slice networks to suit different applications and enterprise needs.

Un des cas d'utilisation est l'infrastructure virtuelle d'entreprise, dans laquelle un portail basé sur SDN peut être étendu selon les particularités de l'organisation. La solution associe la coordination riche d'un contrôleur cloud et d'un contrôleur SDN. Cela permet l'instanciation, la réplication et la migration du réseau et services basés cloud dans la meilleure localisation disponible, en fonction des requis tenants, de la congestion globale du réseau et de la disponibilité de ressources. Cette solution conforme à l'idéal de ne pas limiter le cloud avec les contraintes physiques du data centre, implémentant un suivi de flux et un renforcement de politiques dans un niveau logique pour le cloud. Cela peut englober plusieurs data centres, quelle que soit leur localisation géographique dans l'infrastructure du réseau.
%One such case is the Virtual Enterprise IT Infrastructure – in which an SDN-based gateway can be extended to the enterprise premises. The solution features tight coordination between a feature-rich cloud controller and an SDN controller. This enables the instantiating, replicating and migrating of network and cloud-based services to the best available location, based on the tenant’s requirements, overall network congestion and cloud availability. True to the ideal of not tying cloud services to the constraints of a physical data center, this solution implements flow tracking and policy enforcement at a “logical” cloud level. This encompasses multiple operator data centers, irrespective of their geographic locations and the network infrastructure connecting them.


%Another case is the virtual home gateway. This is an example of virtualizing some of the functions of a traditional home gateway and hosting them in a Network-enabled Cloud. Virtualization reduces the complexity of the home gateway by moving most of the sophisticated functions into the network. As a result, operators can prolong the home gateway refreshment cycle, cut maintenance costs and reduce time to market for new services. The most important aspect of this solution, however, is that it gives the network visibility to all the devices that were traditionally hidden behind the home gateway. This opens up significant revenue opportunities through the ability to offer services that are personalized in a much more granular way.



Un des scénarios les plus traditionnels de l'intégration des services dynamiques avec SDN consiste à en resserrer l'interaction entre le réseau et le cloud. Pour les services "inline" tels que filtrage, modification des entêtes et \gls{nat}, les opérateurs utilisent diverses "appliances", ou d'autres services pour gérer le trafic utilisateur. Ces services sont hébergés dans du matériel physique ou en machines virtuelles. L’enchaînement de services est nécessaire pour router le trafic client à travers ces services. Les solutions traditionnelles sont soit statiques ou très limitées en flexibilité et \gls{scalability}.
%The more traditional and now widely accepted Service Provider SDN use case of dynamic service chaining* itself relies on tight interaction between the network and the cloud. For inline services, such as content filtering, header enrichment, firewalls and Network Address Translation (NAT), operators use different appliances, or value-added services to manage subscriber traffic. These inline services can be hosted on dedicated physical hardware or on virtual machines (software appliances running in a virtualized cloud environment). Service chaining is required to route certain subscriber traffic through more than one such service. Solutions currently available are either static or their flexibility is significantly limited by scalability inefficiencies.

%Dynamic service chaining can optimize the use of extensive high-touch services by either selectively steering traffic through specific services or bypassing them completely. This can provide capex savings through efficient use of capacity. Greater control over traffic and the use of subscriber-based selection of inline services can lead to the creation of new offerings and new ways to monetize networks.

%The Network-enabled Cloud provides the necessary virtual resources for software appliances, whether on dedicated physical hardware or on virtual machines, and supports efficient distribution of these resources wherever needed in the network, such as to best meet latency requirements. 

%Dans un Cloud habilité réseau, l'extension des applications peut être achevée avec la demande de ressources virtuelles. Scaling a software appliance can be achieved either by requesting more cloud capacity in the Network-enabled Cloud or by requesting virtual resources in a centralized cloud data center. The flexibility of the distributed cloud is greatly enhanced using the Service Provider SDN real-time control mechanism, in which software appliances can be moved within or between clouds while preserving the networking attributes and requirements.

Dans le chapitre précédent, les principaux défis des réseaux traditionnels pour fonctionner en mode Cloud ont été détaillés. Un résumé simplifié des principales difficultés a été proposé. Les prochaines sections, traitent ces points et illustrent comment ces objectifs peuvent être atteints en utilisant SDN.


\section{Complexité vs Agilité}

La complexité et l'agilité sont des éléments qui sont fortement accouplés et doivent donc être étudiés ensemble. En effet, la complexité correspond à une importante restriction aux objectifs d'agilité, qui en revanche provoque un niveau plus élevé de complexité dans l'approche réseau traditionnelle. Cet impasse a pu être observé dans les scénarios présentés précédemment. 

Les architectures réseaux implémentant SDN proposent d'aplatir physiquement la topologie avec l'interconnexion de tous les élément à un \gls{fabric} pour à la fois simplifier le réseau et gagner en agilité. L'image ci-dessous visualise cette proposition. Le contrôleur SDN peut être programmé pour reproduire logiquement l'infrastructure souhaitée par tenant. 

\begin{figure}[h]
\begin{center}
\includegraphics[width=0.8\textwidth]{images/RefArchiSDN} 
\caption{Topologie réseau simplifiée. \cite{cloudReadyNetworkJuniper}} \label{RefArchiSDN}
\end{center}
\end{figure} 

Avec les solutions SDN, les changements sur le réseau sont traités par un processus complètement automatisé qui peut réagir instantanément à divers événements. Ainsi que comme on fait pour les VMs, SDN permet de créer des templates de réseaux logiques des tenants qui peuvent être facilement instanciés pour implémenter tous les aspects réseaux requis de manière automatique. 
%With the Nuage Networks solution, changes are handled by a fully automated process that can react instantaneously. This makes network operations much simpler across an open cloud environment.

Les templates peuvent être réutilisés plusieurs fois, si l'opération doit être répétée. Les solutions SDN peuvent créer le réseau nécessaire et fournir l'isolation complète entre tenants, ce qui simplifie considérablement les opérations réseaux au sein des data centres. 
%The Nuage Networks VSP automates the full instantiation of the tenant networking requirements through SDN programmability and abstraction.
%The template can be used as many times as required, should this operation need to be repeated multiple times. The Nuage Networks VSP solution instantiates the necessary networking and provides full isolation between tenants and through the robust implementation of SDN technology pillars, which simplifies the operation of the data center network substantially.

Cela apporte une réponse plus rapide aux demandes des clients avec une simplification opérationnelle. Grâce à sa capacité de programmation et d'abstraction, SDN élimine les opérations manuelles très susceptibles aux erreurs, avec une meilleure efficacité du réseau, réduction de coûts et plus de profit.
%Faster response time to customer demands, which results in much higher customer satisfaction
%• Simplified business operations through templates that are created once and used many times
%• Operational simplification through automation, with SDN programmability and abstraction
%• Elimination of tedious manual operations, which in turn eliminates the potential for human errors associated with implementation and modification of customer tenant configurations
%• Higher data center network and server efficiency through elimination of bottlenecks, ultimately resulting in lower CAPEX and higher profitability

\section{Sécurité}

La sécurité dans un data centre Cloud doit protéger le trafic entre les clients et les serveurs, le trafic entre les machines virtuelles dans les serveurs ainsi que le trafic entre serveurs (physiques ou virtuels) et les applications ou systèmes dans d'autres data centres. La capacité d'extension s'agit d'un requis de sécurité essentiel dans ces environnements. 



%Security administrators in a cloud-ready data center must protect client-to-server traffic, traffic between virtual machines on servers, and traffic between physical and virtual servers, applications, and systems in other data centers . The ability to scale is a primary security requirement in these environments. 

L'importante croissance de l'accès utilisateur ainsi que de la sophistication des menaces de sécurité dans un data centre Cloud exigent une visibilité étendue du réseau et la mise en place des protections associées. Les solutions de sécurité doivent à la fois renforcer les politiques de manière consistante et rester flexibles pour assurer l'adaptation du réseau aux divers usages.

%Increasing user access and the rising sophistication of security threats in a cloud-ready data center also require expanded visibility into threat vectors and related protection . at the same time, security solutions still need to consistently enforce policies, while remaining flexible to adapt to the changes in traffic volumes and data flows that occur because of virtualization, web 2 .0 applications, and cloud services . appropriate policies affect the availability of business critical applications and operations .

Pour aborder ces défis, la sécurité d'un cloud doit réunir des capacités telles que \gls{scalability}, monitoring et contrôles renforcés. Les services de sécurité doivent être consolidés et coordonnés pour complimenter la simplification et la mutualisation agile du réseau. Cette approche améliore la flexibilité et l'efficacité du système entier.
%To address these challenges, capabilities such as scale, visibility, and enforcement controls must work together to comprehensively secure a cloud-ready data center . Security services must be consolidated and pooled in a coordinated fashion to complement the simplification and sharing of the network . This approach enhances the flexibility and efficiency of the entire solution .

SDN propose des moyens pour fournir des services dynamiques de sécurité et pour répondre aux requis de performances tout en accommodant les évolutions futures à la demande. De services tels que supervision, filtrage, détection et prévention d'intrusion et VPNs sont consolidés dans une plateforme extensible avec de ressources affectées dynamiquement. La solution améliore également chaque service de sécurité en augmentant de manière dynamique leur capacité d'accès à tout flux de trafic au sein du \gls{fabric} réseau.
%Big Tap enhances the functionality of each network security and monitoring appliance by dynamically extending its reach to any traffic flow within the network fabric. 
%Juniper networks has developed high- performance, cloud-enabled dynamic security services to meet today’s security and performance requirements, while accommodating future on-demand growth . Services such as application identification and monitoring, stateful firewall, intrusion detection and prevention, and VPns are consolidated on an expandable platform that flexibly and dynamically assigns resources as needed . 

Les services de sécurité doivent être conscients des applications, tout en étant mobiles. L'interface de programmation fournie avec SDN permet l'interaction avec les hyperviseurs en vue d'assurer la sécurité inter-VMs. Les applications peuvent impliquer au déploiement d'un ensemble de politiques de sécurité à partir des hyperviseurs et à travers le \gls{fabric} réseau, grâce à l'établissement d'un lien entre les couches virtuelles et physiques. 
%Security services must be application- and identity-aware, while providing the mobile workforce with secure access to data center applications . Juniper solutions now include integrated and comprehensive vGw Series virtual machine security capabilities for securing virtualized data centers . Businesses can deploy a consistent set of security policies and services from the hypervisor all the way across the network fabric, bridging virtual and physical network layers . 

Comme résultat, les consommateurs peuvent apercevoir la complète de la virtualisation, tout en étant protégés contre les risques de sécurité associés. SDN franchi les deux principales préoccupations concernant les plateformes Cloud : la sécurité et le contrôle. Le réseau, ayant contact à tous les éléments de l'infrastructure, est le meilleur endroit pour placer la sécurité et la gestion. Avec SDN, ces défis peuvent être abordés pour libérer le Cloud Computing. 
%As a result, customers can realize the full value of virtualization, while protecting against the associated security risks .

%The network is the best place to secure and manage the cloud: The two biggest barriers to broader use of cloud computing remains security and control (see Exhibit 2). Many IT managers are unclear as to how to secure and manage resources that they no longer own, and are not on-premises. Pushing control and security points to the network allows IT managers to meet these challenges. The network is the only IT asset that touches every other IT resource.


\addchap{Conclusion}

Même avec le succès incontestable de l'architecture d'internet, l'état de l'industrie réseau et l'essence de son infrastructure se trouvent en phase critique. Il est généralement admis que les réseaux courants sont excessivement chers, compliqués à gérer, sujets aux blocages des fournisseurs et difficiles à faire évoluer. 

On constate donc un réel besoin de faire évoluer cette architecture mais des résistances s'opposent à cette évolution en raison de la complexité et la possible saturation du système. En réponse, les réseaux programmables ont été un objet intensif de recherche par la communauté. Les travaux dans ce domaine s'orientent vers l'offre SDN, un nouveau paradigme transformant cette architecture.

L'approche SDN sépare le plan de contrôle et le plan de données, offrant un contrôle et une vision centralisés du réseau. Cela peut apporter certains bénéfices comme le contrôle directement programmable, la simplification du hardware réseau et la simplification de l'ingénierie du trafic. En revanche, des défis d'implémentation sont à surmonter tels que la concentration des risques dans un contrôle physiquement centralisé, l'équilibre entre flexibilité et performance et les conditions d'interopérabilité.

La flexibilité apportée par SDN est telle que de nombreuses possibilités d'applications sont à imaginer. Essentiellement pour l'administration de data centers, le contrôle d'accès et de la mobilité pour les réseaux campus ainsi que  l'ingénierie du trafic pour les réseaux WAN.

Le marché suit de près les nouveautés dans le domaine et investit sur les technologies implémentant SDN. Les stratégies ne sont pas encore assez matures et les consommateurs potentiels attendent des offres plus consolidées. Cependant, des solutions innovantes commencent à surgir et certaines sociétés assument le rôle de tête dans le marché.

On s'aperçoit que l'ampleur des possibilités SDN, même si elle présente un avantage en théorie, freine son adoption. En raison de la grande variété de concepts et produits, les consommateur hésitent toujours à prendre une décision. En même temps, les grands fournisseurs cherchent à la fois à exploiter le nouveau marché et à protéger leurs solutions consolidées. Ces obstacles même s'ils sont confirmés, ne semblent pas être assez forts pour empêcher les échanges à long terme.

Au vu cette étude, il semblerait que dans un futur proche, les clients les plus informés et les plus disposés à innover vont commencer à déployer SDN. Leurs expériences et les résultats obtenus  vont fortement impacter le choix des prochains consommateurs. Il est possible que  ceux qui dessineront le futur de la technologie des réseaux informatiques pour les prochaines années seront ceux qui auront osé se lancer les premiers. Cette démarche peut éventuellement représenter un risque, mais aussi l'opportunité de tirer des bénéfices plus durables et de prendre de plus larges parts du marché. 

%\appendix
%\include{ann1}
%\include{ann2}
%\gls{sdn}
%\gls{paradigme}
%\gls{ti}

\backmatter
%\nocite{*}

\printindex

\glsaddall

\printglossary[type=acronym,title=Acronymes,toctitle=Acronymes]
\printglossary[type=main,title=Glossaire,toctitle=Glossaire]
%\printglossaries

\printbibliography

\cleardoublepage
\includepdf[pages={3-4}]{couverture-ebt.pdf}
\end{document}
