%\chapter{Apports de SDN aux data centres}
%Ce chapitre démontre les apports de SDN au sein des data centres par rapports aux problématiques présentées précédemment.

\section{Scénarios d'utilisation}

Un système cloud qui s'intègre de façon transparente et dynamique avec un réseau programmable (grâce à SDN) peut fournir une importante plus-value à ses opérateurs et à leurs abonnés (consommateurs finaux et entreprises). Aujourd'hui la connectivité seule ne suffit pas, les utilisateurs réclament une variété de services hébergés dans le cloud, et cela exige des réseau la capacité de fournir la connectivité correcte à l'application souhaitée. Ce dans ce cadre où  la réelle valeur d'un cloud à réseau programmable dynamiquement devient visible.
%A cloud system that integrates seamlessly with a real-time, programmable network – enabled by Service Provider SDN – can provide significant value to network operators and their subscribers (both consumers and enterprises). Today, most subscribers do not rely on connectivity alone. Instead, they demand a wide range of services that are cloud-hosted, and they require the network to play a role in offering the right connectivity for the desired application. This is where the real value of a Service Provider SDN-based, real-time programmable network and cloud becomes apparent.

Cette capacité permet de découper le réseau en tranches et offrir aux clients leurs morceaux dédiées et personnalisés. Il y a une variété des scénarios imaginables  à partir de ce concept de diviser le réseau pour convenir à différents applications et besoins.
%A “meta” use case is the ability to slice and offer consumers/enterprises a piece of the network-plus-cloud for their dedicated, personalized use. There are multiple variants of use cases that are based on this concept of the ability to slice networks to suit different applications and enterprise needs.

Un des cas d'utilisation est l'infrastructure virtuelle entreprise, dans laquelle un portail basé SDN peut être étendu pour les particularité de l'organisation. La solution associe la coordination riche d'un contrôleur et d'un contrôleur SDN. Cela permet l'instanciation, réplication et migration du réseau et services basés cloud dans la meilleure localisation disponible, en fonction des requis tenants, congestion globale du réseau et disponibilité de ressources. Conforme à l'idéal de ne pas limité le cloud avec les contrainte physique du data centre,  cette solution implémente un suivi de flux et renforcement de politiques dans un niveau logique pour le cloud. Cela englobe plusieurs data centres, quelle que soit leur localisation géographique dans l'infrastructure du réseau.
%One such case is the Virtual Enterprise IT Infrastructure – in which an SDN-based gateway can be extended to the enterprise premises. The solution features tight coordination between a feature-rich cloud controller and an SDN controller. This enables the instantiating, replicating and migrating of network and cloud-based services to the best available location, based on the tenant’s requirements, overall network congestion and cloud availability. True to the ideal of not tying cloud services to the constraints of a physical data center, this solution implements flow tracking and policy enforcement at a “logical” cloud level. This encompasses multiple operator data centers, irrespective of their geographic locations and the network infrastructure connecting them.


Another case is the virtual home gateway. This is an example of virtualizing some of the functions of a traditional home gateway and hosting them in a Network-enabled Cloud. Virtualization reduces the complexity of the home gateway by moving most of the sophisticated functions into the network. As a result, operators can prolong the home gateway refreshment cycle, cut maintenance costs and reduce time to market for new services. The most important aspect of this solution, however, is that it gives the network visibility to all the devices that were traditionally hidden behind the home gateway. This opens up significant revenue opportunities through the ability to offer services that are personalized in a much more granular way.

Avec le cloud, SDN et NFV travaillant ensemble, on peut dynamiquement étendre les fonctions réseau dans le cloud. Lors que la charge réseau augmente, le contrôleur SDN peut demander au gestionnaire cloud d'instancier une nouvelle fonction réseau dans le cloud pour commencer à repartir le trafic. (pas claire)
Another case that brings out the value of a combined real-time programmable network and cloud solution is the ability to dynamically extend network functions into the cloud – with SDN, NFV and the cloud all working together. As the load on a network appliance increases, the SDN controller can request a peer cloud manager to instantiate a virtual network function in the cloud and to start load balancing between the physical appliance and the virtual appliance, treating it as a common entity.

Un des scénarios les plus traditionnel de l'intégration des services dynamiques avec SDN consiste en resserrer l'interaction entre le réseau et le cloud. Pour les services inline tels que filtrage, modification des entêtes et \gls{nat}, les opérateurs utilisent divers appliances, ou d'autres services pour gérer le trafic utilisateur. Ces services sont hébergés dans du matériel physique ou en machines virtuelles. L’enchaînement de services est nécessaire pour router le trafic client à travers ce services. Les solutions disponibles actuellement sont soit statique ou très limitées en flexibilité et extension.
%The more traditional and now widely accepted Service Provider SDN use case of dynamic service chaining* itself relies on tight interaction between the network and the cloud. For inline services, such as content filtering, header enrichment, firewalls and Network Address Translation (NAT), operators use different appliances, or value-added services to manage subscriber traffic. These inline services can be hosted on dedicated physical hardware or on virtual machines (software appliances running in a virtualized cloud environment). Service chaining is required to route certain subscriber traffic through more than one such service. Solutions currently available are either static or their flexibility is significantly limited by scalability inefficiencies.

Dynamic service chaining can optimize the use of extensive high-touch services by either selectively steering traffic through specific services or bypassing them completely. This can provide capex savings through efficient use of capacity. Greater control over traffic and the use of subscriber-based selection of inline services can lead to the creation of new offerings and new ways to monetize networks.

The Network-enabled Cloud provides the necessary virtual resources for software appliances, whether on dedicated physical hardware or on virtual machines, and supports efficient distribution of these resources wherever needed in the network, such as to best meet latency requirements. 

Scaling a software appliance can be achieved either by requesting more cloud capacity in the Network-enabled Cloud or by requesting virtual resources in a centralized cloud data center. The flexibility of the distributed cloud is greatly enhanced using the Service Provider SDN real-time control mechanism, in which software appliances can be moved within or between clouds while preserving the networking attributes and requirements.



\section{Complexité}

\section{Agilité}

\section{Sécurité}
