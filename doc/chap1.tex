\chapter{L'évolution des data centres vers le Cloud Computing et le Software-Defined Data Centre}
\label{chap-1}

Ce chapitre a pour but de définir un data centre afin de pouvoir analyser ses problématiques, enjeux et possibles solutions. En vue de comprendre l'état actuel des data centres et ses limitations par rapport aux nouveaux besoins et challenges business. Un regard sur le nouveau business model apporté avec le Cloud Computing, les bénéfices de son adoption et les enjeux pour les infrastructures qui doivent répondre à ce nouveau paradigme.

\section{Data centres et ses objectifs}

Un data centre (ainsi détoné ferme de serveurs) est un répertoire centralisé pour le stockage, management et distribution de données et informations. Typiquement, un data centre est une installation utilisée pour loger des systèmes informatiques et ses composants associés, tels que systèmes de télécommunication et stockage. \cite{understandingCloudWhatDC}

Les data centres traditionnels hébergent typiquement des nombreuses applications relativement petites ou moyennes, chacune exécutant dans une infrastructure matérielle dédiée qui est isolée et protégée des autres systèmes dans la même installation. Ces data centres accueillent du matériel et du logiciel pour multiples unités organisationnelles ou même diverses entreprises. Différents systèmes informatiques au sein d'un tel data centre ont souvent très peu en commun en termes de matériel, logiciel ou infrastructure de maintenance, et en général ne se communiquent pas entre eux. 


Les tendances vers l'informatique côté serveur et l'explosion en popularité des services sur internet ont changé ce scénario. Des infrastructures data centre entières ont été dédiée à un seul acteur pour faire fonctionner ses services offerts. Dans ce cadre, un data centre appartient à une seule organisation et utilise du matériel et plateforme logicielle relativement homogènes qui partagent une couche commune de systèmes de management. Surtout, ces data centres dédiés exécutent un nombre réduit d'applications (ou services internet) beaucoup plus important en taille, l'infrastructure commune de management permettant une significative flexibilité de déploiement. 

Ces infrastructures sont montées pour gérer la taille des applications déployées et la haute disponibilité exigée pour ces services, visant en général 99,99\% de durée de fonctionnement (une heure au maximum de temps d'arrêt par an). Atteindre un fonctionnement libre des failles dans une large collection de systèmes matériel et logiciel dur et devient encore plus difficile avec le grand nombre de serveurs impliqués. Les infrastructures de ces data centres doivent être dimensionnées précisément pour en fonction de la charge des applications supportées. Par conséquence, des nouvelles approches ont été proposées pour la construction et opération de ces systèmes qui doivent être conçus pour tolérer ce nombre important des failles avec très peu ou aucun impact sur la performance et disponibilité des services offerts. \cite{datacenterAsComputerIntro}

\section{Organisation d'un data centre et difficultés}

%A data center is generally organized in rows of ‘‘racks” where each rack contains modular assets such as servers, switches, storage ‘‘bricks”, or specialized appliances
Un data centre est en général organisé en lignes de racks où chaque rack contient des dispositifs modulaires tels que serveurs, switches, briques de stockages ou instruments spécialisés. %Trois principaux éléments d'infrastructure constituent les data centres : le stockage, le réseau et l'approvisionnement énergétique.
Des composants essentiels de l'infrastructure des data centres d'entreprises tels que compute, stockage et réseau sont la base sur laquelle les applications business sont construites qui sont branchés aux racks. Un chassis vient complet avec ses propre ventilateurs, source d'alimentation, panier d'interconnexion et module de management. 
Pour réduire l'espace occupé, des serveurs peuvent être compartimentés dans chassis  L'importante croissance en densité de serveurs, réalisée 

\section{Virtualisation}

Virtualization refers to the abstraction of logical resources away from their underlying physical resources to improve agility and flexibility, reduce costs, and thus enhance business value. Virtualization allows a set of underutilized physical infrastructure components to be consolidated into a smaller number of better utilized devices, contributing to significant cost savings.

Server virtualization is a method of abstracting the operating system from the hardware platform. This allows multiple operating systems or multiple instances of the same operating system to coexist on one or more processors. A hypervisor or virtual machine monitor (VMM) is inserted between the operating system and the hardware to achieve this separation. These operating systems are called “guests” or “guest OSs.” The hypervisor provides hardware emulation to the guest operating systems. It also manages allocation of hardware resources between operating systems.


Integrated infrastructure solutions are specifically designed to provide advantages compared to a conventional physical infrastructure because they are: 
•	 Efficient in power usage, space utilization and IT employee productivity
•	 Economical in initial cost by making use of existing infrastructure and not requiring expensive room upgrades
•	 Interoperable through simplified design and implementation of systems and components 
•	 Controllable through planning, monitoring and management over the changing IT environment

\section{Technologies Associées}

\section{Principaux Challenges}

%According to a recent Gartner study, the leading challenges facing today’s data centers are intrinsic to many of the aforementioned business drivers and their associated IT solutions. Top challenges cited include: 
%• Keeping up with data growth 
%• Maintaining system performance and scalability
%• Mitigating network congestion and connectivity issues
%• Minimizing power, cooling and space costs
%• Effectively managing the data center and its infrastructure
%Furthermore, according to a 2011 survey of the Data Center Users’ Group (DCUG), the leading infrastructure challenges included data center availability, high heat densities, energy efficiency and maintaining adequate power densities (see Figure 1). Each of these challenges resonates closely with the leading data center challenges faced by IT professionals.

\section{Tendances des meilleures pratiques}

\subsection{Énergie}
1. Maximize the return temperature at the cooling units to improve capacity and efficiency
2. Match cooling capacity and airflow with IT loads
3. Utilize cooling designs that reduce energy consumption
4. Select a power system to optimize your availability and efficiency needs
5. Design for flexibility using scalable architectures that minimizes footprint

%Integrated infrastructure solutions are specifically designed to provide advantages compared to a conventional physical infrastructure because they are: 
%•	 Efficient in power usage, space utilization and IT employee productivity
%•	 Economical in initial cost by making use of existing infrastructure and not requiring expensive room upgrades
%•	 Interoperable through simplified design and implementation of systems and components 
%•	 Controllable through planning, monitoring and management over the changing IT environment
\section{Architecture}

\section{Cloud Computing}

Information technology (IT) is at a breaking point, and there is a critical need to improve IT's impact on the business.9
Consider the following:
 -As much as 85\% of computing capacity sits idle in distributed computing environments.
 -Seventy percent of IT budgets is typically spent on maintaining current IT infrastructures, and only 30\% is typically spent on new capabilities.
 -Over 30\% of consumers notified of a security breach will terminate their relationship with the company that contributed to the breach.
Clearly, infrastructures need to be more dynamic to free up budgets for new investments and accelerate deployment of superior capabilities being demanded by the business. Nearly all CEOs are adapting business models; cloud adoption can support these changing business dynamics.