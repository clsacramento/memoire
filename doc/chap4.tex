
\chapter{Applications SDN et ses apports aux data centres}

Ce chapitre redéfinira SDN et présentera ses réponses aux problématiques réseau rencontrées en général dans les data centres, qui ont été débattues dans le chapitre précédent. %Comment SDN approche la problématique ? Qu'est-ce que SDN ?
Le chapitre démontrera également les apports de SDN au sein des data centres par rapport aux problématiques présentées précédemment.

\section{Redéfinition de SDN}

The definition of network services must be application-centric and simple. The networking requirements of applications should be articulated in an IT-friendly language and used to request desired network behaviors. Today there is limited programmability, resulting in an unfortunate tradeoff. Engineering of cloud applications is either encumbered by detailed network-specific parameters and information, or ignores them and treats the network as a “black box”. Neither is optimal. The first option leads to custom applications for each network type while with the second, applications lack the visibility and control needed to leverage the true capabilities of the network.
The proper abstraction of network capabilities, consistent with Software Defined Networking (SDN) principles, drives programmability of network services and increases the consumability of network resources.
The instantiation of network services must be instantaneous, aligned with the needs of applications. The network should act reflexively to establish the required connectivity within and across datacenters, and do so in a manner that is consistent with the defined policies of CSPs and their tenants. Today that process is slow, manual and error-prone, relying on work orders to establish a multitude of vendor- dependent configurations.



\section{Solutions}

Nuage NetworksTM removes the constraints of the datacenter network through an innovative Virtualized Services Platform (VSP) that abstracts network capabilities and automates service instantiation. With the Nuage Networks Software Defined Networking solution, cloud service providers, web-scale operators and large tech enterprises can b uild a robust and scalable multi-tenant networking infrastructure that delivers secure virtual slices of readily consumable compute, storage and networking instantaneously across thousands of tenants and user groups.

\section{SDN et le Cloud Computing}