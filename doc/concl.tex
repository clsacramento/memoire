\addchap{Conclusion}

%Même avec le succès incontestable de l'architecture d'internet, l'état de l'industrie réseau et l'essence de son infrastructure se trouvent en phase critique. Il est généralement admis que les réseaux courants sont excessivement chers, compliqués à gérer, sujets aux blocages des fournisseurs et difficiles à faire évoluer. 

Au sein des data centres, les technologies de virtualisation des serveurs ont évolué considérablement pour accompagner les nouveaux besoins clients. De nouvelles technologies pour la virtualisation du stockage se trouvent également disponible sur le marché. Toutefois, la réelle valeur du Cloud Computing ne pourra pas être atteinte sans une évolution similaire dans l'aspect réseau.

On constate donc un réel besoin de faire évoluer les technologies réseau mais des résistances s'opposent à cette évolution en raison de la complexité et la possible saturation du système. Avec l'architecture réseau traditionnelle, les opérateurs cloud ont des difficultés à l'adapter au rythme actuel des demandes et à assurer le niveau de sécurité exigé.

Des contraintes de complexité opérationnelle et de sécurité sur les réseaux Cloud empêchent le déploiement de nouveaux services et applications à l'agilité demandée. De cette manière, les réseaux deviennent cibles de critiques, étant accusés de freiner le rythme d'innovation espéré aujourd'hui.

En réponse, les réseaux programmables ont été un objet intensif de recherche par la communauté. Les travaux dans ce domaine s'orientent vers la virtualisation du réseau (NFV) facilitée par l'offre SDN, un nouveau paradigme transformant l'architecture traditionnelle. 

L'approche SDN sépare le plan de contrôle et le plan de données, offrant un contrôle et une vision centralisés du réseau. Cela peut apporter certains bénéfices comme le contrôle directement programmable, la simplification des équipements et de l'ingénierie du trafic. %En revanche, des défis d'implémentation sont à surmonter tels que la concentration des risques dans un contrôle physiquement centralisé, l'équilibre entre flexibilité et performance et les conditions d'interopérabilité.


Le contrôleur SDN, en association avec l'orchestration cloud, permet d'approvisionner les réseaux dynamiquement et de manière simplifié, tout en assurant la sécurité nécessaire. Même si l'approche est encore jeune, elle est suivi attentivement par le marché qui accompagne la parution des premières offres SDN, proposées par de vendeurs majeurs ainsi que par des startups.
%La flexibilité apportée par SDN est telle que de nombreuses possibilités d'applications sont à imaginer. Essentiellement pour l'administration de data centers, le contrôle d'accès et de la mobilité pour les réseaux campus ainsi que  l'ingénierie du trafic pour les réseaux WAN.


Bien que la technologie se développe à un rythme accéléré, on attend sa consolidation pour une adoption plus large. Le marché bouge avec des acteurs proposant des solutions stratégiquement différentes selon leurs produits de base et le consommateur final hésite à ne pas choisir la bonne.
%Le marché suit de près les nouveautés dans le domaine et investit sur les technologies implémentant SDN. Les stratégies ne sont pas encore assez matures et les consommateurs potentiels attendent des offres plus consolidées. Cependant, des solutions innovantes commencent à surgir et certaines sociétés assument le rôle de tête dans le marché.

Entre-temps, le besoin pour une infrastructure plus agile, intégrée par le cloud, continue à augmenter et les fournisseurs ont intérêt à réagir rapidement dans la bataille pour des parts de marché. Ceux qui dessineront le futur de la technologie des réseaux informatiques pour les prochaines années seront ceux qui auront osé se lancer les premiers.
%On s'aperçoit que l'ampleur des possibilités SDN, même si elle présente un avantage en théorie, freine son adoption. En raison de la grande variété de concepts et produits, les consommateur hésitent toujours à prendre une décision. En même temps, les grands fournisseurs cherchent à la fois à exploiter le nouveau marché et à protéger leurs solutions consolidées. Ces obstacles même s'ils sont confirmés, ne semblent pas être assez forts pour empêcher les échanges à long terme.

%Au vu cette étude, il semblerait que dans un futur proche, les clients les plus informés et les plus disposés à innover vont commencer à déployer SDN. Leurs expériences et les résultats obtenus  vont fortement impacter le choix des prochains consommateurs. Il est possible que  ceux qui dessineront le futur de la technologie des réseaux informatiques pour les prochaines années seront ceux qui auront osé se lancer les premiers. Cette démarche peut éventuellement représenter un risque, mais aussi l'opportunité de tirer des bénéfices plus durables et de prendre de plus larges parts du marché. 